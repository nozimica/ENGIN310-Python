\newcommand\tikzmark[2][]{%
  \tikz[remember picture,overlay,baseline=-.5ex] \node[#1] (#2) {};}

%%%%%%%%%%%%%%%%%%%%%%%%%%%%%%%%%%%%%%%%%%%%%%%%%%%%%%%%%%%%%%%%%%%%%%%%%%%%%%
%
% \braceDownwards<>[styles]{firstNode}{secondNode}
%%%%%%%%%%%%%%%%%%%%%%%%%%%%%%%%%%%%%%%%%%%%%%%%%%%%%%%%%%%%%%%%%%%%%%%%%%%%%%
% http://tex.stackexchange.com/questions/1559/adding-a-large-brace-next-to-a-body-of-text/1570#1570
% http://tex.stackexchange.com/questions/204417/how-can-i-draw-arrows-within-a-listing-to-explain-positional-association
% https://tex.stackexchange.com/questions/110072/tikz-inside-lstlisting-inside-tikz
%%%%%%%%%%%%%%%%%%%%%%%%%%%%%%%%%%%%%%%%%%%%%%%%%%%%%%%%%%%%%%%%%%%%%%%%%%%%%%
\newcommand<>\braceDownwards[3][ultra thick,structure!80,font=\footnotesize]{%
  \tikz[remember picture,overlay] \draw [decoration={brace,raise=3pt,amplitude=0.5em},decorate,#1]
        (#2.east) -- node[below=3pt] (#2#3) {} (#3.west);
}
%%%%%%%%%%%%%%%%%%%%%%%%%%%%%%%%%%%%%%%%%%%%%%%%%%%%%%%%%%%%%%%%%%%%%%%%%%%%%%
%
% \braceRightwards<>[styles]{firstNode}{secondNode}
\newcommand<>\braceRightwards[3][ultra thick,structure!80,font=\footnotesize]{%
  \tikz[remember picture,overlay] \draw [decoration={brace,raise=17em,amplitude=0.5em},decorate,#1]
        (#2.north) -- node[right=17em] (#2#3) {} (#3.south);
}

%%%%%%%%%%%%%%%%%%%%%%%%%%%%%%%%%%%%%%%%%%%%%%%%%%%%%%%%%%%%%%%%%%%%%%%%%%%%%%
%
% \braceDownwardsLabel<>[styles]{firstNode}{secondNode}{distToLabel}{labelText}{labelWidth}
% \braceDownwardsLabel{entradaB}{entradaA}{6mm}{las entradas}{6em}
\newcommand<>\braceDownwardsLabel[6][ultra thick,structure!80,font=\footnotesize]{%
  \braceDownwards[#1]{#2}{#3}
  \begin{tikzpicture}[remember picture, overlay]
    \node (text#2#3) [below=#4 of #2#3,text centered,text width=#6,#1] {#5};
    \draw[->,#1] (#2#3) -- (text#2#3);
  \end{tikzpicture}
}
%%%%%%%%%%%%%%%%%%%%%%%%%%%%%%%%%%%%%%%%%%%%%%%%%%%%%%%%%%%%%%%%%%%%%%%%%%%%%%
%
% \braceRightwardsLabel<>[styles]{firstNode}{secondNode}{distToLabel}{labelText}{labelWidth}
% \braceRightwardsLabel{entradaB}{entradaA}{6mm}{las entradas}{6em}
\newcommand<>\braceRightwardsLabel[6][ultra thick,structure!80,font=\footnotesize]{%
  \braceRightwards[#1]{#2}{#3}
  \begin{tikzpicture}[remember picture, overlay]
    \node (text#2#3) [right=#4 of #2#3,text centered,text width=#6,#1] {#5};
    \draw[->,#1] (#2#3) -- (text#2#3);
  \end{tikzpicture}
}


\newcommand{\muestraRecetaDisenno}{%
    \bgnblockdefinition[wd=.8\textwidth,centered]
        {\large \bld{Receta de diseño:}}%
        \begin{enumerate}%
            \item Explicar su propósito.%
            \item Dar ejemplos de su uso.%
            \item Especificar el cuerpo de la función.%
            \item Comprobar que hace lo correcto.%
        \end{enumerate}%
    \trmblockdefinition
}

% http://tex.stackexchange.com/questions/75337/tikz-arrow-outside-tikzpicture-environment/75343#75343
\newcommand*{\DrawArrow}[3][]{%
    % #1 = draw options
    % #2 = left point
    % #3 = right point
    \begin{tikzpicture}[overlay,remember picture]
        \draw [-latex, #1] ($(#2)+(0.1em,0.5ex)$) to ($(#3)+(0,0.5ex)$);
    \end{tikzpicture}%
}%


%% BEGIN @
\makeatletter
%%%%%%%%%%%%%%%%%%%%%%%%%%%%%%%%%%%%%%%%%%%%%%%%%%%%%%%%%%%%%%%%%%%%%%%%%%%%%%
% 
% \drawBrace[styles]{beginningNode}{endingNode}{comment}
%
% This command draws a brace between two tikzmarks, to their right,
% no matter which one is the rightmost, and includes 
% a node midway the brace, to write the comment.
% This command also creates a new node
% whose name is the concat of the names of beginning and ending nodes.
% Into styles there can be styles for: decoration, brace and comment.
\pgfkeys{/nzmacros/drawBrace/.cd,%
    decoration/.initial={},%
    brace/.initial={},%
    comment/.initial={},%
}
\newcommand<>{\drawBrace}[4][]{%
  \begin{tikzpicture}[remember picture, overlay]
  \pgfkeys{/nzmacros/drawBrace/.cd,#1}
  \node[draw=none, fit={(#2) (#3)}, inner sep=0pt] (rectg) {};%
  \draw [decoration={brace,amplitude=0.4em}, decorate, very thick, red, /nzmacros/drawBrace/decoration]%
    ([/nzmacros/drawBrace/brace]rectg.north east) --%
    coordinate[right=1em, midway] (#2#3)%
    ([/nzmacros/drawBrace/brace]rectg.south east);%
  \node[right=0em of #2#3, structure, font=\small, /nzmacros/drawBrace/comment] (#2#3-comment) {#4};%
  % \draw (#2#3-comment.west) edge (#2#3);%
  \end{tikzpicture}
}%
\makeatother
%% END @

%%%%%%%%%%%%%%%%%%%%%%%%%%%%%%%%%%%%%%%%%%%%%%%%%%%%%%%%%%%%%%%%%%%%%%%%%%%%%%
% 
% \drawTikZComment[nodeStyle]{codeMark}{message}
%
% Draws a comment pointing to codeMark. Generates a node with message as text,
% and codeMark-comment as name.
\tikzstyle{commentNode} = [text width=35mm, align=center, minimum height=10pt, font=\footnotesize\bfseries, fill=PaleGreen1!60, draw=PaleGreen2, line width=1pt, rounded corners]
\newcommand<>{\drawTikZComment}[3][]{%
  \tikz[remember picture, overlay]%
    \node[commentNode, right=4em of #2, #1] (#2-comment) {#3};%
  \drawTikZAnotherArrow{#2}{#2-comment}%
}%

%%%%%%%%%%%%%%%%%%%%%%%%%%%%%%%%%%%%%%%%%%%%%%%%%%%%%%%%%%%%%%%%%%%%%%%%%%%%%%
% 
% \drawTikZAnotherArrow{codeMark}{messageNode}
%
% From an already drawn comment node called messageNode, draws
% another arrow pointing to codeMark.
\tikzstyle{commentEdge} = [->, thick, >=stealth, gray!70, dashed, line width = 1pt]
\newcommand<>{\drawTikZAnotherArrow}[2]{%
  \tikz[remember picture, overlay]%
    \draw[commentEdge] (#2) -- ($(#1.center)!.6em!(#2)$);%
}%


%%%%%%%%%%%%%%%%%%%%%%%%%%%%%%%%%%%%%%%%%%%%%%%%%%%%%%%%%%%%%%%%%%%%%%%%%%%%%%
% 
% \fullrule
%
% Draws a rule with 'structure' color, spanning the whole width of the text.
\newcommand<>{\fullrule}{%
  \textcolor{structure}{\noindent\rule{\textwidth}{1pt}}%
}


%%%%%%%%%%%%%%%%%%%%%%%%%%%%%%%%%%%%%%%%%%%%%%%%%%%%%%%%%%%%%%%%%%%%%%%%%%%%%%
%%%% Beamer aux commands
%%%%%%%%%%%%%%%%%%%%%%%%%%%%%%%%%%%%%%%%%%%%%%%%%%%%%%%%%%%%%%%%%%%%%%%%%%%%%%

%%%%%%%%%%%%%%%%%%%%%%%%%%%%%%%%%%%%%%%%%%%%%%%%%%%%%%%%%%%%%%%%%%%%%%%%%%%%%%
% 
% \strongText<>{text}
%
% Highlights the text with both bold and blue color.
\newcommand<>{\strongText}[1]{%
    \textcolor#2{blue}{\textbf{#1}}%
}

%%%%%%%%%%%%%%%%%%%%%%%%%%%%%%%%%%%%%%%%%%%%%%%%%%%%%%%%%%%%%%%%%%%%%%%%%%%%%%
% 
% \coordText<3->{node1}{textLeft}{node2}{textRight}
%
% Allows to join a node1 with text to a node2 with text
% using an arrow.
\newcommand<>{\coordText}[4]{%
    \tikz[na] \node (#1) {#2};%
    \begin{tikzpicture}[remember picture,overlay]%
        \node#5[right=35mm of #1.west] (#3) {#4};%
        \path[arrowInList]#5 ([xshift=25mm]#1.west) -- (#3);%
    \end{tikzpicture}%
}

%%%%%%%%%%%%%%%%%%%%%%%%%%%%%%%%%%%%%%%%%%%%%%%%%%%%%%%%%%%%%%%%%%%%%%%%%%%%%%
% 
% \placeLogo<>[width]{logoPath}
%
% Places a logo in the upper right corner of the slide.
% It's a good idea to use it as the last command of the slide.
\newcommand<>{\placeLogo}[2][.2\textwidth]{%
    \tikz[remember picture, overlay]#3 \node[anchor=north east] at (current page.north east)
        {\includegraphics[width=#1,valign=t]{#2}};
}
%%%%%%%%%%%%%%%%%%%%%%%%%%%%%%%%%%%%%%%%%%%%%%%%%%%%%%%%%%%%%%%%%%%%%%%%%%%%%%
% 
% \placeLogoOpt<>[width]{nodeOptions}{logoPath}
%
% Places a logo in the upper right corner of the slide.
% With its second argument it is possible to pass options directly to
% the node holding the logo (such as [x|y]shifts.
% It's a good idea to use it as the last command of the slide.
\newcommand<>{\placeLogoOpt}[3][.2\textwidth]{%
    \tikz[remember picture, overlay]#4 \node[anchor=north east,#2] at (current page.north east)
        {\includegraphics[width=#1,valign=t]{#3}};
}
