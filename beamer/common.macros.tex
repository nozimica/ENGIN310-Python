% \newcommand\tikzmark[2][]{%
%   \tikz[remember picture,overlay,baseline=-.5ex] \node[#1] (#2) {};}

%% BEGIN @
\makeatletter

% http://tex.stackexchange.com/questions/75337/tikz-arrow-outside-tikzpicture-environment/75343#75343
\newcommand*{\DrawArrow}[3][]{%
    % #1 = draw options
    % #2 = left point
    % #3 = right point
    \begin{tikzpicture}[overlay,remember picture]
        \draw [-latex, #1] ($(#2)+(0.1em,0.5ex)$) to ($(#3)+(0,0.5ex)$);
    \end{tikzpicture}%
}%

%%%%%%%%%%%%%%%%%%%%%%%%%%%%%%%%%%%%%%%%%%%%%%%%%%%%%%%%%%%%%%%%%%%%%%%%%%%%%%
%%%% Brace Library
%%%%%%%%%%%%%%%%%%%%%%%%%%%%%%%%%%%%%%%%%%%%%%%%%%%%%%%%%%%%%%%%%%%%%%%%%%%%%%
%
% These commands draw braces between two tikzmarks,
% no matter which one is the rightmost, and include
% a node midway the brace, to write a comment.
% This command also creates a new node
% whose name is the concat of the names of beginning and ending nodes.
% Into styles there can be styles for: decoration, brace and comment.
%
% Based on some tips from these QAs:
% http://tex.stackexchange.com/questions/1559/adding-a-large-brace-next-to-a-body-of-text/1570#1570
% http://tex.stackexchange.com/questions/204417/how-can-i-draw-arrows-within-a-listing-to-explain-positional-association
% https://tex.stackexchange.com/questions/110072/tikz-inside-lstlisting-inside-tikz
%%%%%%%%%%%%%%%%%%%%%%%%%%%%%%%%%%%%%%%%%%%%%%%%%%%%%%%%%%%%%%%%%%%%%%%%%%%%%%

%%%%%%%%%%%%%%%%%%%%%%%%%%%%%%%%%%%%%%%%%%%%%%%%%%%%%%%%%%%%%%%%%%%%%%%%%%%%%%
%
% \nzupdatepgfkeys{keysFromMacro}
%
% Set default values for pgfkeys, and load new keys from macro.
% Example: \nzupdatepgfkeys{#1}
\newcommand\nzupdatepgfkeys[1]{%
    \pgfkeys{/nzmacros/drawBrace/.cd,%
        decoration/.style={structure},%
        brace/.style={},%
        braceAmplitude/.initial={0.4em},%
        labelDistance/.initial=1em,%
        labelWidth/.initial=6em,%
    }
    \pgfkeys{/nzmacros/drawBrace/.cd,#1}
}
%%%%%%%%%%%%%%%%%%%%%%%%%%%%%%%%%%%%%%%%%%%%%%%%%%%%%%%%%%%%%%%%%%%%%%%%%%%%%%
%
% \braceDownwards<>[styles]{firstNode}{secondNode}
\newcommand<>\braceDownwards[3][]{%
  \nzupdatepgfkeys{#1}
  \begin{tikzpicture}[remember picture, overlay]
    \node[fit={(pic cs:#2) (pic cs:#3)}, inner sep=0pt] (rectg) {};%
    \drawBraceInner{rectg.south east}{rectg.south west}{#2#3}
  \end{tikzpicture}
}

%%%%%%%%%%%%%%%%%%%%%%%%%%%%%%%%%%%%%%%%%%%%%%%%%%%%%%%%%%%%%%%%%%%%%%%%%%%%%%
%
% \braceDownwardsLabel<>[styles]{firstNode}{secondNode}{labelText}
\newcommand<>\braceDownwardsLabel[4][]{%
  \nzupdatepgfkeys{#1}
  \braceDownwards[#1]{#2}{#3}
  \begin{tikzpicture}[remember picture, overlay]
      \node[below=\pgfkeysvalueof{/nzmacros/drawBrace/labelDistance} of #2#3, structure, font=\small, text width=\pgfkeysvalueof{/nzmacros/drawBrace/labelWidth}, text centered] (#2#3-comment) {#4};%
  \end{tikzpicture}
}

%%%%%%%%%%%%%%%%%%%%%%%%%%%%%%%%%%%%%%%%%%%%%%%%%%%%%%%%%%%%%%%%%%%%%%%%%%%%%%
%
% \braceUpwards<>[styles]{firstNode}{secondNode}
\newcommand<>{\braceUpwards}[3][]{%
  \nzupdatepgfkeys{#1}
  \begin{tikzpicture}[remember picture, overlay]
    \node[fit={($(pic cs:#2)+(0,1ex)$) ($(pic cs:#3)+(0,1ex)$)}, inner sep=0pt] (rectg) {};%
    \drawBraceInner{rectg.north west}{rectg.north east}{#2#3}
  \end{tikzpicture}
}%
%%%%%%%%%%%%%%%%%%%%%%%%%%%%%%%%%%%%%%%%%%%%%%%%%%%%%%%%%%%%%%%%%%%%%%%%%%%%%%
%
% \braceRightwards<>[styles]{firstNode}{secondNode}
\newcommand<>\braceRightwards[3][]{%
  \nzupdatepgfkeys{#1}
  \begin{tikzpicture}[remember picture, overlay]
    \node[fit={($(pic cs:#2)+(0,1ex)$) (pic cs:#3)}, inner sep=0pt] (rectg) {};%
    \drawBraceInner{rectg.north east}{rectg.south east}{#2#3}
  \end{tikzpicture}
}

%%%%%%%%%%%%%%%%%%%%%%%%%%%%%%%%%%%%%%%%%%%%%%%%%%%%%%%%%%%%%%%%%%%%%%%%%%%%%%
%
% \braceRightwardsLabel<>[styles]{firstNode}{secondNode}{labelText}
\newcommand<>\braceRightwardsLabel[4][]{%
  \nzupdatepgfkeys{#1}
  \braceRightwards[#1]{#2}{#3}
  \begin{tikzpicture}[remember picture, overlay]
      \node[right=\pgfkeysvalueof{/nzmacros/drawBrace/labelDistance} of #2#3, structure, font=\small, text width=\pgfkeysvalueof{/nzmacros/drawBrace/labelWidth}] (#2#3-comment) {#4};%
    % \draw (#2#3-comment.west) edge (#2#3);%
  \end{tikzpicture}
}

%%%%%%%%%%%%%%%%%%%%%%%%%%%%%%%%%%%%%%%%%%%%%%%%%%%%%%%%%%%%%%%%%%%%%%%%%%%%%%
% 
% \drawBraceInner{leftNode}{rightNode}{middlePoint}
%
% Draws a brace from leftNode to rightNode, putting a coordinate at
% the middle named middlePoint.
\newcommand<>{\drawBraceInner}[3]{%
  \draw [decoration={brace,amplitude=\pgfkeysvalueof{/nzmacros/drawBrace/braceAmplitude}, raise=1mm}, decorate, very thick, /nzmacros/drawBrace/decoration]%
    ([/nzmacros/drawBrace/brace]#1) -- coordinate[midway] (#3) ([/nzmacros/drawBrace/brace]#2);%
}
\makeatother
%% END @

%%%%%%%%%%%%%%%%%%%%%%%%%%%%%%%%%%%%%%%%%%%%%%%%%%%%%%%%%%%%%%%%%%%%%%%%%%%%%%
%%%% Comments Library
%%%%%%%%%%%%%%%%%%%%%%%%%%%%%%%%%%%%%%%%%%%%%%%%%%%%%%%%%%%%%%%%%%%%%%%%%%%%%%
%
% Some macros to draw comments pointing to a tikzmark.
%%%%%%%%%%%%%%%%%%%%%%%%%%%%%%%%%%%%%%%%%%%%%%%%%%%%%%%%%%%%%%%%%%%%%%%%%%%%%%

%%%%%%%%%%%%%%%%%%%%%%%%%%%%%%%%%%%%%%%%%%%%%%%%%%%%%%%%%%%%%%%%%%%%%%%%%%%%%%
% 
% \drawTikZComment[nodeStyle]{codeMark}{message}
%
% Draws a comment pointing to codeMark. Generates a node with message as text,
% and codeMark-comment as name.
\tikzstyle{commentNode} = [text width=35mm, align=center, minimum height=10pt, font=\footnotesize\bfseries, fill=PaleGreen1!60, draw=PaleGreen2, line width=1pt, rounded corners]
\newcommand<>{\drawTikZComment}[3][]{%
  \tikz[remember picture, overlay]%
    \node[commentNode, right=4em of pic cs:#2, #1] (#2-comment) {#3};%
  \drawTikZAnotherArrow{#2}{#2-comment}%
}%

%%%%%%%%%%%%%%%%%%%%%%%%%%%%%%%%%%%%%%%%%%%%%%%%%%%%%%%%%%%%%%%%%%%%%%%%%%%%%%
% 
% \drawTikZAnotherArrow{codeMark}{messageNode}
%
% From an already drawn comment node called messageNode, draws
% another arrow pointing to codeMark.
\tikzstyle{commentEdge} = [->, thick, >=stealth, gray!70, dashed, line width = 1pt]
\newcommand<>{\drawTikZAnotherArrow}[2]{%
  \tikz[remember picture, overlay]%
    \draw[commentEdge] (#2) -- ($($(pic cs:#1)+(0,.5ex)$)!.6em!(#2)$);%
    % \draw[commentEdge] (#2) -- ($(pic cs:#1.center)!.6em!(#2)$);%
}%


%%%%%%%%%%%%%%%%%%%%%%%%%%%%%%%%%%%%%%%%%%%%%%%%%%%%%%%%%%%%%%%%%%%%%%%%%%%%%%
%%%% Beamer aux commands
%%%%%%%%%%%%%%%%%%%%%%%%%%%%%%%%%%%%%%%%%%%%%%%%%%%%%%%%%%%%%%%%%%%%%%%%%%%%%%

%%%%%%%%%%%%%%%%%%%%%%%%%%%%%%%%%%%%%%%%%%%%%%%%%%%%%%%%%%%%%%%%%%%%%%%%%%%%%%
% 
% \strongText<>{text}
%
% Highlights the text with both bold and blue color.
\newcommand<>{\strongText}[1]{%
    \textcolor#2{blue}{\textbf{#1}}%
}

%%%%%%%%%%%%%%%%%%%%%%%%%%%%%%%%%%%%%%%%%%%%%%%%%%%%%%%%%%%%%%%%%%%%%%%%%%%%%%
% 
% \coordText<3->{node1}{textLeft}{node2}{textRight}
%
% Allows to join a node1 with text to a node2 with text
% using an arrow.
\newcommand<>{\coordText}[4]{%
    \tikz[na] \node (#1) {#2};%
    \begin{tikzpicture}[remember picture,overlay]%
        \node#5[right=35mm of #1.west] (#3) {#4};%
        \path[arrowInList]#5 ([xshift=25mm]#1.west) -- (#3);%
    \end{tikzpicture}%
}

%%%%%%%%%%%%%%%%%%%%%%%%%%%%%%%%%%%%%%%%%%%%%%%%%%%%%%%%%%%%%%%%%%%%%%%%%%%%%%
% 
% \fullrule
%
% Draws a rule with 'structure' color, spanning the whole width of the text.
\newcommand<>{\fullrule}{%
  \textcolor{structure}{\noindent\rule{\textwidth}{1pt}}%
}

%%%%%%%%%%%%%%%%%%%%%%%%%%%%%%%%%%%%%%%%%%%%%%%%%%%%%%%%%%%%%%%%%%%%%%%%%%%%%%
% 
% \placeLogo<>[width]{logoPath}
%
% Places a logo in the upper right corner of the slide.
% It's a good idea to use it as the last command of the slide.
\newcommand<>{\placeLogo}[2][.2\textwidth]{%
    \tikz[remember picture, overlay]#3 \node[anchor=north east] at (current page.north east)
        {\includegraphics[width=#1,valign=t]{#2}};
}
%%%%%%%%%%%%%%%%%%%%%%%%%%%%%%%%%%%%%%%%%%%%%%%%%%%%%%%%%%%%%%%%%%%%%%%%%%%%%%
% 
% \placeLogoOpt<>[width]{nodeOptions}{logoPath}
%
% Places a logo in the upper right corner of the slide.
% With its second argument it is possible to pass options directly to
% the node holding the logo (such as [x|y]shifts.
% It's a good idea to use it as the last command of the slide.
\newcommand<>{\placeLogoOpt}[3][.2\textwidth]{%
    \tikz[remember picture, overlay]#4 \node[anchor=north east,#2] at (current page.north east)
        {\includegraphics[width=#1,valign=t]{#3}};
}

\newcommand{\muestraRecetaDisenno}{%
    \bgnblockdefinition[wd=.8\textwidth,centered]
        {\large \bld{Receta de diseño:}}%
        \begin{enumerate}%
            \item Explicar su propósito.%
            \item Dar ejemplos de su uso.%
            \item Especificar el cuerpo de la función.%
            \item Comprobar que hace lo correcto.%
        \end{enumerate}%
    \trmblockdefinition
}

