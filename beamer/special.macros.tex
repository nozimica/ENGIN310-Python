%% BEGIN @
\makeatletter
\newlength{\nzSignMargin}

\newif\if@nzsign@turnleft%
\newif\if@nzsign@turnright%
\newif\if@nzsign@keepstraight%
\pgfkeys{/nzsign/.cd,%
    turnleft/.is if=@nzsign@turnleft,%
    turnright/.is if=@nzsign@turnright,%
    keepstraight/.is if=@nzsign@keepstraight,%
}
\newcommand<>{\tikzSign}[4][]{%
  \node (#3) at (#4) {};%
  \pgfkeys{/nzsign/.cd,#1}%
  \setlength{\nzSignMargin}{#2}%
  \draw[rounded corners, yellow, fill=yellow] (#3.center) -- ++(#2,#2) -- ++(-#2,#2) -- ++(-#2,-#2) -- cycle;%
  \tikzstyle{signFrame} = [rounded corners, line width=.05*\the\nzSignMargin]%
  \tikzstyle{signLine} = [line width=.15*\the\nzSignMargin]%
  \tikzstyle{signArrow} = [-{Stealth[length=.3*\the\nzSignMargin,inset=.05*\the\nzSignMargin,width=.3*\the\nzSignMargin]}]%
%
  % \addtolength{\nzSignMargin}{-.1#2}
  \draw[signFrame] ($(#3.center)+(0,.1\nzSignMargin)$) -- ++(.9\nzSignMargin,.9\nzSignMargin) -- ++(-.9\nzSignMargin,.9\nzSignMargin) -- ++(-.9\nzSignMargin,-.9\nzSignMargin) -- cycle;%
  \draw[signLine] ($(#3.center)+(0,.4\nzSignMargin)$) -- ($(#3.center)+(0,.75\nzSignMargin)$);%
  \if@nzsign@turnleft%
    \draw[signLine, signArrow] ($(#3.center)+(0,.75\nzSignMargin)$) to[out=90,in=-30] ($(#3.center)+(-.5\nzSignMargin,1.3\nzSignMargin)$);%
  \fi%
  \if@nzsign@turnright%
    \draw[signLine, signArrow] ($(#3.center)+(0,.75\nzSignMargin)$) to[out=90,in=-150] ($(#3.center)+(+.5\nzSignMargin,1.3\nzSignMargin)$);%
  \fi%
  \if@nzsign@keepstraight%
    \draw[signLine, signArrow] ($(#3.center)+(0,.75\nzSignMargin)$) -- ($(#3.center)+(0,1.7\nzSignMargin)$);%
  \fi%
  \pgfkeys{/nzsign/.cd,turnleft=false,turnright=false,keepstraight=false}%
}

%%%%%%%%%%%%%%%%%%%%%%%%%%%%%%%%%%%%%%%%%%%%%%%%%%%%%%%%%%%%%%%%%%%%%%%%%%%%%%
% 
% \placeLogoTurnSignBig<>{elements}
%
% Places a logo with turn signs. What branches are drawn is specified
% in the parameter elements. This is a big logo.
% Example: \placeLogoTurnSignBig{keepstraight,turnleft,turnright}
\newcommand<>{\placeLogoTurnSignBig}[1]{%
    \tikz[remember picture, overlay]#2 \tikzSign[#1]{12mm}{a1}{$(current page.north east)+(-16mm,-28mm)$};%
    \vspace{-3ex}%
}

%%%%%%%%%%%%%%%%%%%%%%%%%%%%%%%%%%%%%%%%%%%%%%%%%%%%%%%%%%%%%%%%%%%%%%%%%%%%%%
% 
% \placeLogoTurnSignSmall<>{elements}
%
% Places a logo with turn signs. What branches are drawn is specified
% in the parameter elements. This is a somewhat small logo.
% Example: \placeLogoTurnSignSmall{keepstraight,turnleft,turnright}
\newcommand<>{\placeLogoTurnSignSmall}[1]{%
    \tikz[remember picture, overlay]#2 \tikzSign[#1]{8mm}{a1}{$(current page.north east)+(-16mm,-20mm)$};%
    \vspace{-3ex}%
}

%%%%%%%%%%%%%%%%%%%%%%%%%%%%%%%%%%%%%%%%%%%%%%%%%%%%%%%%%%%%%%%%%%%%%%%%%%%%%%
%%%% Specific diagram macros
%%%%%%%%%%%%%%%%%%%%%%%%%%%%%%%%%%%%%%%%%%%%%%%%%%%%%%%%%%%%%%%%%%%%%%%%%%%%%%
% 
% \nzThreeStepsSolving<>{elements}
%
% For 1.01algoritmos
% Draws three blocks, with opacity in those not selected.
% Example: \nzThreeStepsSolving{blockA=1}
\pgfkeys{/nzmacros/threestepssolving/.cd,%
    blockA/.initial=0.3,%
    blockB/.initial=0.3,%
    blockC/.initial=0.3,%
}
\newcommand<>{\nzThreeStepsSolving}[1]{%
  \pgfkeys{/nzmacros/threestepssolving/.cd,#1,%
    blockA/.get=\nzmacros@threestepssolving@blockA,%
    blockB/.get=\nzmacros@threestepssolving@blockB,%
    blockC/.get=\nzmacros@threestepssolving@blockC,%
  }
  \vspace{1em}
  \begin{small}
  \begin{tikzpicture}[node distance=2cm]
    \node (a1) [simplestep, opacity=\nzmacros@threestepssolving@blockA] {Identificar el problema};
    \node (a2) [simplestep, opacity=\nzmacros@threestepssolving@blockB, right=10mm of a1] {Diseñar un algoritmo};
    \node (a3) [simplestep, opacity=\nzmacros@threestepssolving@blockC, right=10mm of a2] {Programar la solución};
    \draw [arrow] (a1) -- (a2);
    \draw [arrow] (a2) -- (a3);
  \end{tikzpicture}
  \end{small}
}

\tikzstyle{simplestep} = [normalfig, rectangle, text width=7em, bottom color=orange!30, draw=orange!90!black!50, rounded corners=2mm]
\tikzstyle{na} = [baseline=-0.5ex,remember picture]
\tikzstyle{arrowInList} = [draw=blue,thick,->]

\makeatother
%% END @
