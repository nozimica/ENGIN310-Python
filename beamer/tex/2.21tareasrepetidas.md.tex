## Realizando cálculos {.fragile}

Hagamos algunos cálculos simples:

- Si una milla equivale a aprox. 1,609344 kilómetros.

\vspace*{-1em}
\bgncolumns

\column{.5\textwidth}

\bgnblocknormal[wd=.7\textwidth,centered=true]
Transformar 6 millas a kilómetros
\trmblocknormal

\column{.5\textwidth}
\vspace{-2ex}
\begin{lstlisting}[linebackgroundcolor={\btLstHL<3->{2-3}}]
>>> distMillas = 6
>>> kmMi = 1.609344
>>> distKm = distMillas * kmMi
>>> distKm
9.656064
\end{lstlisting}
\trmcolumns

\pause

\vspace{1ex}
\bgncolumns
\column{.5\textwidth}

\bgnblocknormal[wd=.7\textwidth,centered=true]
Transformar 0,3 millas a kilómetros
\trmblocknormal

\column{.5\textwidth}
\vspace{-2ex}
\begin{lstlisting}[linebackgroundcolor={\btLstHL<3->{2-3}}]
>>> distMillas = 0.3
>>> kmMi = 1.609344
>>> distKm = distMillas * kmMi
>>> distKm
0.4828032
\end{lstlisting}

\trmcolumns

\pause
\bgnblockdanger
OJO: ¡¡se repite casi todo en ambos cálculos!!
\trmblockdanger

## Realizando cálculos {.fragile}

O por ejemplo, podríamos querer calcular el área de un círculo.

\bgncolumns

\column{0.5\textwidth}

\bgnblocknormal[wd=.7\textwidth,centered=true]
¿Cuánto vale el área de un círculo de radio 10 cm?
\trmblocknormal

\column{0.5\textwidth}
\vspace{-2ex}
\begin{lstlisting}[linebackgroundcolor={\btLstHL<3->{2-3}}]
>>> radio = 10
>>> PI = 3.1415
>>> area = PI * (radio**2)
>>> area
314.15
\end{lstlisting}

\trmcolumns

\pause

\vspace{1ex}
\bgncolumns

\column{0.5\textwidth}

\bgnblocknormal[wd=.7\textwidth,centered=true]
¿Cuánto vale el área de un círculo de radio 300 cm?
\trmblocknormal

\column{0.5\textwidth}
\vspace{-2ex}
\begin{lstlisting}[linebackgroundcolor={\btLstHL<3->{2-3}}]
>>> radio = 300
>>> PI = 3.1415
>>> area = PI * (radio**2)
>>> area
282735.0
\end{lstlisting}

\trmcolumns

\pause

\bgnblockdanger
¡¡ también se repite casi todo !!
\trmblockdanger

## Funciones {.fragile}

\bgnblocknormal[wd=.8\textwidth,centered=true]
\bld{Funciones:} agrupando tareas que se repiten.
\trmblocknormal

\bgncolumns

\column{.5\textwidth}

\begin{lstlisting}
>>> distMillas = 6
>>> kmMi = 1.609344
>>> distKm = distMillas * kmMi
>>> distKm
9.656064
\end{lstlisting}

\column{.5\textwidth}

\begin{lstlisting}[style=frame01]
>>> # Transforma un valor desde millas a kilómetros.
>>> def mi2Km(distMillas):
        kmMi = 1.609344
        return distMillas * kmMi
\end{lstlisting}

\trmcolumns

\pause

- Una vez definidas, las funciones pueden ser usadas:

\bgncolumns

\column{.35\textwidth}

\begin{lstlisting}
>>> mi2Km(6)
9.656064
>>> mi2Km(0.3)
0.4828032
\end{lstlisting}

\column{.65\textwidth}

\bgnblocknormal[wd=.9\textwidth,centered=true]
Las usamos tal como lo hemos hecho en matemática.
\trmblocknormal

\trmcolumns

## Funciones {.fragile}

\bgnblocknormal[wd=.8\textwidth,centered=true]
\bld{Funciones:} agrupando tareas que se repiten.
\trmblocknormal

\bgncolumns

\column{.5\textwidth}

\begin{lstlisting}
>>> radio = 10
>>> PI = 3.1415
>>> area = PI * (radio**2)
>>> area
314.15
\end{lstlisting}

\column{.5\textwidth}

\begin{lstlisting}[style=frame01]
>>> # Calcula el área de un círculo, dado su radio.
>>> def areaCirculo(radio):
        PI = 3.1415
        return PI * (radio ** 2)
\end{lstlisting}

\trmcolumns

\pause

- Una vez definidas, las funciones pueden ser usadas:

\bgncolumns

\column{.35\textwidth}
\begin{lstlisting}
>>> areaCirculo(10)
314.15000000000003
>>> areaCirculo(2)
12.566
\end{lstlisting}

\column{.65\textwidth}

\bgnblocknormal[wd=.9\textwidth,centered=true]
Las usamos tal como lo hemos hecho en matemática.
\trmblocknormal

\trmcolumns

## Funciones {.fragile}

Formalicemos el concepto de \strongText{Función}:

\vfill
\bgnblockdefinition
\bld{Función:} Corresponde a un conjunto de instrucciones que realizan una tarea concreta.
Puede recibir datos de entrada y puede generar salidas y/o mensajes.
\trmblockdefinition

\vfill

Nos concentraremos en escribir funciones que:

- Hagan una tarea muy específica.
- Tengan definidas sus entradas y salidas.
- Tengan bien establecido qué es lo que hace, mediante comentarios justo arriba de
su definición.

\placeLogoOpt{xshift=-5mm,yshift=-1mm}{img/factory-1f3ed.pdf}

## Realizando cálculos (cont.) {.fragile}

Ahora realicemos el siguiente cálculo:

- Un automóvil rinde 14 km/lt.
- Sabemos que recorrió 24 millas.
- Queremos saber cuántos litros de combustible consumió.

\vspace{-.6ex}
\bgnblocknormal
Podríamos hacer los cálculos paso a paso...
\trmblocknormal

\begin{onlyenv}<2->

\bgncolumns
\column{.5\textwidth}

\begin{lstlisting}[linebackgroundcolor={%
        \btLstHL<3->{3-4}%
}]
>>> rendKmLt = 14
>>> distMillas = 24
>>> kmMi = 1.609344
>>> distKm = distMillas * kmMi
>>> consumoLt = distKm / rendKmLt
>>> consumoLt
2.7588754285714288
\end{lstlisting}

\column{.05\textwidth}

\column{.45\textwidth}<3->

\bgnblockalert
\footnotesize Pero eso ya está hecho, es convertir millas a km:
\trmblockalert


\begin{lstlisting}[linebackgroundcolor={ \btLstHL{1-3} }]
def mi2Km(distMillas):
	kmMi = 1.609344
	return distMillas * kmMi
\end{lstlisting}


\trmcolumns

\end{onlyenv}

\placeLogoOpt{xshift=-15mm,yshift=-6mm}{img/truck-1f69a.pdf}

## Realizando cálculos (cont.) {.fragile}

- Mirémoslo nuevamente:

\vspace{-2ex}

\bgncolumns
\column{.48\textwidth}
\begin{lstlisting}[linebackgroundcolor={%
        \btLstHL{3-4}%
}]
>>> rendKmLt = 14
>>> distMillas = 24
>>> kmMi = 1.609344
>>> distKm = distMillas * kmMi
>>> consumoLt = distKm / rendKmLt
>>> consumoLt
2.7588754285714288
\end{lstlisting}

\column{.04\textwidth}

\column{.48\textwidth}

\begin{lstlisting}[linebackgroundcolor={%
        \btLstHL{2-4}%
}]
# Transforma un valor desde millas a kilómetros.
def mi2Km(distMillas):
    kmMi = 1.609344
    return distMillas * kmMi
\end{lstlisting}

\trmcolumns

\pause

- Todo eso se transforma en:

\vspace{-2ex}

\bgncolumns
\column{.2\textwidth}
\column{.6\textwidth}
\begin{lstlisting}[linebackgroundcolor={%
        \btLstHL{3}%
}]
>>> rendKmLt = 14
>>> distMillas = 24
>>> distKm = mi2Km(distMillas)
>>> consumoLt = distKm / rendKmLt
>>> consumoLt
2.7588754285714288
\end{lstlisting}

\column{.2\textwidth}

\trmcolumns

## Realizando cálculos (cont.) {.fragile}

\bgnblocknormal

- ¿Qué pasa si cambia el rendimiento?
- ¿Qué pasa si cambian las millas recorridas?

\trmblocknormal

\bgncolumns
\column{.48\textwidth}
\begin{lstlisting}[linebackgroundcolor={%
        \btLstHL{1-2}%
}]
>>> rendKmLt = 14
>>> distMillas = 24
>>> distKm = mi2Km(distMillas)
>>> consumoLt = distKm / rendKmLt
>>> consumoLt
2.7588754285714288
\end{lstlisting}

\column{.04\textwidth}
\column{.48\textwidth}
\begin{lstlisting}[linebackgroundcolor={\btLstHL{1-2}}]
>>> rendKmLt = 17
>>> distMillas = 12
>>> distKm = mi2Km(distMillas)
>>> consumoLt = distKm / rendKmLt
>>> consumoLt
1.1360075294117649
\end{lstlisting}

\trmcolumns

\bgncolumns
\column{.48\textwidth}

- Ojo: esas dos líneas son las \textbf{entradas} de un cálculo que se repite:

\column{.04\textwidth}

\column{.48\textwidth}

\bgnblocknormal
Calcular el consumo de litros, en función del rendimiento y lo recorrido.
\trmblocknormal

\trmcolumns

## Realizando cálculos (cont.) {.fragile}

\vspace*{-3ex}
\bgncolumns

\column{0.4\textwidth}

\bgnblocknormal
Para este código definimos una nueva función: \texttt{consumoVehiculo}
\trmblocknormal

\column{0.05\textwidth}
\column{0.55\textwidth}

\begin{lstlisting}
>>> rendKmLt = 14
>>> distMillas = 24
>>> distKm = mi2Km(distMillas)
>>> consumoLt = distKm / rendKmLt
\end{lstlisting}

\trmcolumns

\vfill

- Noten que esta función recibe \strongText{dos} entradas.

\vfill

\begin{lstlisting}[style=frame01]
>>> # Calcula el consumo de combustible en litros, para un vehículo
>>> # dado su rendimiento [km/lt] y la distancia recorrida [mi].
>>> def consumoVehiculo(rendKmLt, distMillas):
>>>     distKm = mi2Km(distMillas)
>>>     consumoLt = distKm / rendKmLt
>>>     return consumoLt
\end{lstlisting}

