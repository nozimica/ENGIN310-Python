## Abstracción

- Ya hemos hablado muchas veces de enfocarnos siempre en:

\bgnblockgood
\structure{Soluciones genéricas}.
\trmblockgood

- En los 5 problemas sobre listas recursivas que vimos hace poco:
    - Al buscar un lugar en una lista de cartas:
        - Encontramos la solución para cualquier lugar dado.
        - No sólo programar la búsqueda de ``isla gaviota'' en particular.

    - O cuando buscábamos las cartas que nos harían excavar una determinada cantidad de metros:
        - No pensar específicamente en 4 metros.

    - Al ``esquivar'' un lugar en una lista:
        - No pensar específicamente en ``playa larga''.

    - Y así, sucesivamente.

## Abstracción {.fragile}

- ¿Qué pasa si hubiésemos programado específicamente una función para buscar
``isla gaviota''?

\begin{lstlisting}[style=frame02]
def buscarIslaGaviota(vLista):
    if vLista == listaVacia:        # caso base
        return False
    else:                           # caso recursivo
        cp = cabeza(vLista)
        if cp.lugar == "isla gaviota":
            return True
        else:
            resto = cola(vLista)
            return buscarIslaGaviota(resto)
\end{lstlisting}

\bgnblockdanger
¡Esto sólo sirve para un caso en particular!
\trmblockdanger

## Abstracción {.fragile}

\bgncolumns
\column{0.45\textwidth}

- ¿Y si después necesitamos buscar ``isla de pascua''?

\begin{lstlisting}[style=frame03]
def buscarIslaDePascua(vLista):
  if vLista == listaVacia:
    return False
  else:
    cp = cabeza(vLista)
    if cp.lugar == "isla de pascua":
      return True
    else:
      resto = cola(vLista)
      return buscarIslaDePascua(resto)
\end{lstlisting}

\column{0.1\textwidth}
\column{0.45\textwidth}

- Comparémosla con la función anterior (para ``isla gaviota''):

\begin{lstlisting}[style=frame03]
def buscarIslaGaviota(vLista):
  if vLista == listaVacia:
    return False
  else:
    cp = cabeza(vLista)
    if cp.lugar == "isla gaviota":
      return True
    else:
      resto = cola(vLista)
      return buscarIslaGaviota(resto)
\end{lstlisting}
\trmcolumns

## Abstracción {.fragile}

\bgncolumns
\column{0.45\textwidth}
\vspace{-2ex}

\begin{lstlisting}[style=frame03]
def buscarIslaDePascua(vLista):
  if vLista == listaVacia:
    return False
  else:
    cp = cabeza(vLista)
    if cp.lugar == "isla de pascua":
      return True
    else:
      resto = cola(vLista)
      return buscarIslaDePascua(resto)
\end{lstlisting}

\column{0.1\textwidth}
\column{0.45\textwidth}
\vspace{-2ex}

\begin{lstlisting}[style=frame03]
def buscarIslaGaviota(vLista):
  if vLista == listaVacia:
    return False
  else:
    cp = cabeza(vLista)
    if cp.lugar == "isla gaviota":
      return True
    else:
      resto = cola(vLista)
      return buscarIslaGaviota(resto)
\end{lstlisting}
\trmcolumns

\bgncolumns
\column{0.2\textwidth}
\column{0.6\textwidth}

\begin{lstlisting}[style=frame03]
def buscarLugar(vLista, buscado):
    if vLista == listaVacia:
        return False
    else:
        cp = cabeza(vLista)
        if cp.lugar == buscado:
            return True
        else:
            resto = cola(vLista)
            return buscarLugar(resto, buscado)
\end{lstlisting}

\column{0.2\textwidth}
\trmcolumns

## Abstracción: comparaciones específicas {.fragile}

- En el ejemplo anterior, la clave fue crear el parámetro \ttt{buscado}.
- Este parámetro se \bld{compara} con el valor de cada elemento de la lista.
    - Esta comparación es el test de igualdad: \ttt{==}

\bgnblockidea
No siempre será tan simple como usar \ttt{==}
\trmblockidea


## Abstracción: comparaciones específicas {.fragile}

\simpleTitle{Ejemplo:}

- Tenemos un listado de las ventas del día de ayer (en CLP)
para las sucursales de una cadena de minimarkets.

- Queremos programar tres búsquedas:
    1. Cuántas sucursales vendieron \$0 (estuvieron cerradas quizás).
    1. Cuántas sucursales vendieron menos de \$1.000.000.
    1. Cuántas sucursales vendieron más de \$15.000.000.

- Supondremos estos datos están representados en una lista de \ttt{int}.

\bgnblockidea
Tenemos que basarnos ahora en otros de los ejemplos que vimos con listas:
\structure{Contar elementos que cumplan una condición} y 
\structure{Búsqueda condicionada}
\trmblockidea

## Abstracción: comparaciones específicas {.fragile}

\simpleTitle{¿Cuántas sucursales vendieron \$0?}

\begin{lstlisting}[style=frame02]
def ventasCero(vLista):
    if vLista == listaVacia:
        return 0
    else:
        cp = cabeza(vLista)
        resto = cola(vLista)
        if cp == 0:
            return 1 + ventasCero(resto)
        else:
            return ventasCero(resto)
\end{lstlisting}

## Abstracción: comparaciones específicas {.fragile}

\simpleTitle{¿Cuántas sucursales vendieron menos de \$1.000.000?}

\begin{lstlisting}[style=frame02]
def ventasMenorMM(vLista):
    if vLista == listaVacia:
        return 0
    else:
        cp = cabeza(vLista)
        resto = cola(vLista)
        if cp < 1000000:
            return 1 + ventasMenorMM(resto)
        else:
            return ventasMenorMM(resto)
\end{lstlisting}

## Abstracción: comparaciones específicas {.fragile}

\simpleTitle{¿Cuántas sucursales vendieron más de \$15.000.000?}

\begin{lstlisting}[style=frame02]
def ventasMayor15MM(vLista):
    if vLista == listaVacia:
        return 0
    else:
        cp = cabeza(vLista)
        resto = cola(vLista)
        if cp > 15000000:
            return 1 + ventasMayor15MM(resto)
        else:
            return ventasMayor15MM(resto)
\end{lstlisting}


## Abstracción: comparaciones específicas {.fragile}

Ahora comparemos estas tres soluciones:

\vspace{-2ex}

\bgncolumns
\column{0.48\textwidth}

\begin{lstlisting}[style=frame03]
def ventasCero(vLista):
  if vLista == listaVacia:
    return 0
  else:
    cp = cabeza(vLista)
    resto = cola(vLista)
    if cp == 0:
      return 1 + ventasCero(resto)
    else:
      return ventasCero(resto)
\end{lstlisting}

\column{0.04\textwidth}
\column{0.48\textwidth}

\begin{lstlisting}[style=frame03]
def ventasMenorMM(vLista):
  if vLista == listaVacia:
    return 0
  else:
    cp = cabeza(vLista)
    resto = cola(vLista)
    if cp < 1000000:
      return 1 + ventasMenorMM(resto)
    else:
      return ventasMenorMM(resto)
\end{lstlisting}

\trmcolumns

\vspace{-2ex}

\bgncolumns
\column{0.48\textwidth}

\begin{lstlisting}[style=frame03]
def ventasMayor15MM(vLista):
  if vLista == listaVacia:
    return 0
  else:
    cp = cabeza(vLista)
    resto = cola(vLista)
    if cp > 15000000:
      return 1 + ventasMayor15MM(resto)
    else:
      return ventasMayor15MM(resto)
\end{lstlisting}

\column{0.04\textwidth}
\column{0.48\textwidth}

\pause

\vspace{2ex}
\bgnblockgood

- La diferencia está en el nombre de la función.
- Y en la línea que hace la comparación.

\trmblockgood

\trmcolumns

## Abstracción: comparaciones específicas {.fragile}

- En vez de estas tres funciones, ¿qué tal si hacemos sólo una, y que junto con la lista
reciba alguna señal que le permita saber qué comparación realizar?

\pause

\bgnblockidea
Una solución: entregar como segundo parámetro una herramienta que represente el \bld{criterio}
de comparación.
\trmblockidea

## Abstracción: comparaciones específicas {.fragile}

- Una manera de hacerlo en Python es:
    1. \bld{\structure{Criterios}}: Por cada criterio existente, crear una función que implemente esa comparación.
        - Esta función retorna un boolean.
    1. \bld{\structure{Función genérica}}: Se entrega esta función de comparación a nuestra función abstracta y genérica de búsqueda de ventas.

## Abstracción: implementar criterios {.fragile}

- He aquí las funciones que implementan cada uno de los tres criterios que estamos
analizando:

\begin{lstlisting}[style=frame02]
def igualCero(valor):
    return valor == 0

def menor1MM(valor):
    return valor < 1000000

def mayor15MM(valor):
    return valor > 15000000
\end{lstlisting}

## Abstracción: implementar función genérica {.fragile}

- Esta es la función genérica, que recibe una función criterio, y la aplica:

\begin{lstlisting}[style=frame02]
def ventasPorCriterio(vLista, criterio):
    if vLista == listaVacia:
        return 0
    else:
        cp = cabeza(vLista)
        resto = cola(vLista)
        if criterio(cp):
            return 1 + ventasPorCriterio(resto, criterio)
        else:
            return ventasPorCriterio(resto, criterio)
\end{lstlisting}

## Abstracción: otro ejemplo {.fragile}

- Tenemos una lista cuyos elementos son estructuras \ttt{fraccion}.
- Dada esta lista, queremos determinar si otra fracción ---o su equivalente--- se encuentra
dentro de la lista.

\pause

\simpleTitle{Solución:}

- Ya sabemos cómo comparar una fracción con otra:
    - Comparamos sus 2 componentes (numerador y denominador) entre sí.
- Pero: ¿cómo podemos determinar que dos fracciones son equivalentes?.
    - Sabemos que $\frac{1}{2}$ es equivalente a $\frac{3}{6}$, porque:

    $$ \frac{n_1}{d_1} \quad \text{es equivalente a} \quad \frac{n_2}{d_2} \iff n_1 * d_2 = d_1 * n_2 $$

## Abstracción: otro ejemplo {.fragile}

- Podríamos programar este criterio de comparación directamente en nuestra función de búsqueda.

\pause

- Pero qué pasa si tenemos otra lista, que ahora puede contener \itt{strings}.
    - ¿Podríamos usar la misma función de arriba para esta lista?

\pause

- Sí: usando una sola función de búsqueda y sendas funciones criterio por cada criterio
de comparación.


