## Listas recursivas

\simpleTitle{La búsqueda del tesoro}

- El famoso corsario Barba Azul, tras décadas de saqueos, decidió guardar
    su tesoro. Como buen corsario, era desconfiado, así que ideó una manera de
    hacer largo el proceso de encontrar ese tesoro.

- Justo antes de morir le entregó una carta a su hijo Barba Ploma, con la ubicación del tesoro.
    Este hijo, ambicioso como era, se fue de inmediato a buscarlo.

- Al llegar a esta ubicación, no encontró un tesoro... sino que otra carta con una nueva ubicación.

- Al llegar a esta ubicación, no encontró un tesoro... sino que otra carta con una nueva ubicación.

- ... y así, sucesivamente ...

- Hasta que finalmente sí llegó a una ubicación donde estaba el tesoro.

## Listas recursivas

\simpleTitle{Barba Azul y Python:}

- Conceptualmente, lo que hizo Barba Azul para esconder su tesoro fue una \strongText{Lista Recursiva}:
    - \bld{Caso recursivo:} Ubicación donde hay una carta, con una nueva ubicación.
    - \bld{Caso base:} Ubicación donde no hay carta, sino que el tesoro mismo.

\vfill
\pause

\bgnblockgood
¿Y cómo recorremos el camino al tesoro?
\trmblockgood

1. Se toma la primera carta.
1. Se va a la ubicación indicada.
1. Si está el tesoro, terminamos.
1. Si hay otra carta, volvemos al paso 2.

## Listas recursivas {.fragile}

\bgnblockdefinition
\strongText{Lista Recursiva:} Es un grupo ordenado de estructuras, donde cada estructura está compuesta de
dos elementos principales: \bld{información} y una \bld{referencia al siguiente elemento de la lista (de haberlo)}.
\trmblockdefinition

\pause

- Por lo tanto, en nuestro ejemplo:
    - En cada ``información'' sólo puede haber 1 de estas 2 cosas:
        1. Una carta con la siguiente ubicación.
        1. Ninguna carta (lo que implica que se llegó al tesoro).

## Listas recursivas {.fragile}

- Por ejemplo, podríamos tener esta secuencia:

\bgnblocknormal
\centering \ttt{(carta (carta (carta (carta () ) ) ) ) }
\trmblocknormal

\vfill

- Lo que se puede ilustrar de la siguiente manera:

\centering \includegraphics[width=.8\textwidth]{img/treasureQueue.pdf}

\vfill

- En este caso, la equis representa el haber llegado al final de la lista.
    - Lo que equivale a haber encontrado el tesoro.

## Listas recursivas: valores compuestos {.fragile}

- Primero que todo: ¿cómo representamos las cartas?

\pause

\bgnblockidea
¡¡ Usando estructuras !!
\trmblockidea

\pause

- ¿cómo la representaríamos?

\pause

\begin{lstlisting}[style=frame02]
import estructura
estructura.crear("carta", "lugar referencia profundidad")
\end{lstlisting}

\pause

- Ejemplos:

\begin{lstlisting}[style=frame02]
carta1 = carta("isla gaviota", "detrás de la roca", "2 metros")
carta2 = carta("playa larga", "al terminar el sendero", "3 metros")
carta3 = carta("arrecife escondido", "al medio", "4 metros")
\end{lstlisting}
## Listas recursivas: operaciones {.fragile}

\bgnblockidea
Ahora necesitamos ponernos de acuerdo en un estándar que nos permita usar las listas.
\trmblockidea

\pause

\vfill

\simpleTitle[2ex]{Operaciones sobre listas recursivas:}

- Para una lista recursiva \structure{llamada \ttt{cartas}}, tenemos:

\begin{lstlisting}
carta = cabeza(cartas)  # entrega la primera carta
resto = cola(cartas)    # entrega el resto de la lista,
                        # sin la primera carta
listaVacia == cartas    # es True si la lista está vacía
vacia(cartas)           # es True si la lista está vacía
\end{lstlisting}

\pause

\bgnblockdanger
¿Pero de dónde sacamos estas funciones?
\trmblockdanger

## Listas recursivas: operaciones {.fragile}

\bgnblockidea
Usaremos el módulo \structure{\ttt{lista.py}}
\trmblockidea

\pause

- Este es un módulo provisto por nuestro curso, no está disponible nativamente en Python.
- Ya está disponible en DocenciaWeb.
    - Ahora está en un archivo .zip, junto con el módulo \structure{\ttt{estructura.py}}.
- Dejan el archivo \ttt{lista.py} en el mismo directorio donde escriban el programa que lo usará.

- Nos provee de todas las funciones necesarias para crear y manipular listas.

## Listas recursivas: operaciones {.fragile}

\simpleTitle{¿Cómo lo cargamos?}

\begin{lstlisting}[style=frame02]
>>> from lista import *
\end{lstlisting}

- Se hace solamente una vez en cada programa.

- Fíjense que al importarlo, no se hace de la misma manera que con \ttt{estructura.py}.

\pause

- ¿Por qué esa diferencia?

\pause

- Supongamos que:
    - Tenemos una lista llamada \nzinlinecode{cartas}.
    - En \ttt{lista.py} hay una función llamada \nzinlinecode{vacia}.

\bgncolumns
\column{.4\textwidth}

\begin{lstlisting}[style=frame02]
>>> from lista import *
>>> vacia(cartas)
\end{lstlisting}

\column{.2\textwidth}
\column{.4\textwidth}

\begin{lstlisting}[style=frame02]
>>> import lista
>>> lista.vacia(cartas)
\end{lstlisting}

\trmcolumns


## Listas recursivas: primer ejemplo {.fragile}

\simpleTitle{¿Cuántas ubicaciones visitó Barba Ploma?}

- Supongamos nuevamente que tenemos la variable \ttt{cartas}.

\begin{lstlisting}[style=frame02]
from lista import *

def largoLista(vLista):
    if listaVacia == vLista:    # caso base: no hay más elementos
        return 0
    else:                       # caso recursivo: 1 + largo del resto
        resto = cola(vLista)
        return 1 + largoLista(resto)
    
\end{lstlisting}

## Listas recursivas: construcción {.fragile}

\bgnblockidea
Para construir nuevas listas, usamos la función \ttt{crearLista}.
\trmblockidea

- Como son listas recursivas, veamos cómo construirlas desde el caso base en adelante:
    - Lista sin elementos:\newline
        \nzinlinecodeSM{L0 = listaVacia}
    - Lista con 1 elemento (2 opciones):\newline
        \nzinlinecodeSM{L1 = crearLista(carta1, L0)}\newline
        \nzinlinecodeSM{L1 = crearLista(carta1, listaVacia)}
    - Lista con 2 elementos (3 opciones):\newline
        \nzinlinecodeSM{L2 = crearLista(carta2, L1)}\newline
        \nzinlinecodeSM{L2 = crearLista(carta2, crearLista(carta1, L0))}\newline
        \hbox{\nzinlinecodeSM{L2 = crearLista(carta2, crearLista(carta1, listaVacia))}}
    - Y así, sucesivamente...


## Listas recursivas: manipulación {.fragile}

- Por ejemplo, podríamos entonces construir una lista con 3 elementos:

\begin{lstlisting}[style=frame02]
>>> cartas = crearLista(carta3, L2)
\end{lstlisting}

- Podríamos obtener la primera carta:

\begin{lstlisting}[style=frame02]
>>> cabeza(cartas)    # retorna carta3
\end{lstlisting}

- Y podríamos obtener el resto de la lista

\begin{lstlisting}[style=frame02]
>>> cola(cartas)      # retorna la lista desde carta2 en adelante
\end{lstlisting}

## Listas recursivas: manipulación {.fragile}

\simpleTitle{Código más completo:}

\begin{lstlisting}[style=frame03]
import estructura
from lista import *

estructura.crear("carta", "lugar referencia profundidad")

carta1 = carta("isla gaviota", "detrás de la roca", "2 metros")
carta2 = carta("playa larga", "al terminar el sendero", "3 metros")
carta3 = carta("arrecife escondido", "al medio", "4 metros")

L0 = listaVacia
L1 = crearLista(carta1, L0)
L2 = crearLista(carta2, L1)
cartas = crearLista(carta3, L2)

print("Carta 1:", carta1)
print("Carta 2:", carta2)
print("Carta 3:", carta3)
print("Lista:", cartas)
\end{lstlisting}


## Listas recursivas: comenzando a usarlas {.fragile}

\simpleTitle{Preguntas útiles de responder:}

1. ¿Hay alguna carta que nos lleve a la ``isla gaviota''?
1. ¿Cuántas cartas nos harán excavar 4 metros de profundidad?
1. Haga una nueva lista, pero esquivando ``playa larga''.
1. ¿Cuántos lugares distintos hay en toda la lista?
1. ¿Cuál es la profundidad más repetida en toda la lista?

## Listas recursivas: buscar un valor {.fragile}

\simpleTitle{1.- ¿Hay alguna carta que nos lleve a la ``isla gaviota''?}

\vspace{1ex}
\bld{Estrategia de resolución:}

- Si la lista está vacía:
    - Retorno \ttt{False}
- En caso contrario:
    - Obtengo la \nzinlinecode{cabeza} de la lista.
    - Si su atributo \ttt{lugar}  es ``isla gaviota'':
        - Retorno \ttt{True}
    - En caso contrario:
        - Obtengo la \nzinlinecode{cola} de la lista.
        - Busco recursivamente en esa \nzinlinecode{cola}.

## Listas recursivas: buscar un valor {.fragile}

\simpleTitle{1.- ¿Hay alguna carta que nos lleve a la ``isla gaviota''?}

\bgnblockidea
\bld{Pensemos en el caso genérico:}
    ¿Qué habría que cambiar en lo anterior, para que funcione para un lugar
    cualquiera, no sólo para ``isla gaviota''?
\trmblockidea

\pause

\begin{lstlisting}[style=frame02]
def buscarLugar(vLista, buscado):
    if vLista == listaVacia:        # caso base
        return False
    else:                           # caso recursivo
        estaCarta = cabeza(vLista)
        if estaCarta.lugar == buscado:
            return True
        else:
            resto = cola(vLista)
            return buscarLugar(resto, buscado)
\end{lstlisting}

## Listas recursivas: buscar y contar {.fragile}

\simpleTitle{2.- ¿Cuántas cartas nos harán excavar 4 metros de profundidad?}

\vspace{1ex}
\bld{Estrategia de resolución:}

- Si la lista está vacía:
    - Retorno cero.
- En caso contrario:
    - Obtengo la \nzinlinecode{cabeza} de la lista.
    - Si su atributo \ttt{profundidad}  es "4 metros":
        - Retorno 1 más la llamada recursiva de esta función, aplicada sobre la \nzinlinecode{cola} de esta lista.
    - En caso contrario:
        - Retorno sólo la llamada recursiva de esta función, aplicada sobre la \nzinlinecode{cola} de esta lista.

## Listas recursivas: buscar y contar {.fragile}

\simpleTitle{2.- ¿Cuántas cartas nos harán excavar 4 metros de profundidad?}

\begin{lstlisting}[style=frame02]
def contarProfundidad(vLista, profBusc):
    if vLista == listaVacia:        # caso base
        return 0
    else:                           # caso recursivo
        estaCarta = cabeza(vLista)
        if estaCarta.profundidad == profBusc:
            # retorno 1 más el llamado recursivo
            # la cola la obtengo justo al llamar a la función
            return 1 + contarProfundidad(cola(vLista), profBusc)
        else:
            # la cola la obtengo justo al llamar a la función
            return contarProfundidad(cola(vLista), profBusc)
\end{lstlisting}

## Listas recursivas: buscar y contar {.fragile}

\simpleTitle{2.- ¿Cuántas cartas nos harán excavar 4 metros de profundidad?}

Busquen la diferencia entre este código y el anterior. Ambos son equivalentes.

\begin{lstlisting}[style=frame02]
def contarProfundidad(vLista, profBusc):
    if vLista == listaVacia:        # caso base
        return 0
    else:                           # caso recursivo
        if cabeza(vLista).profundidad == profBusc:
            # retorno 1 más el llamado recursivo
            # la cola la obtengo justo al llamar a la función
            return 1 + contarProfundidad(cola(vLista), profBusc)
        else:
            # la cola la obtengo justo al llamar a la función
            return contarProfundidad(cola(vLista), profBusc)
\end{lstlisting}


## Listas recursivas: buscar y descartar {.fragile}

\simpleTitle{3.- Haga una nueva lista, pero esquivando ``playa larga''}

\vspace{1ex}
\bld{Estrategia de resolución:}

\bgnblockidea
\bld{Ojo:} vamos a crear una nueva lista (es decir, una copia) de la lista
original, con la salvedad que cada vez que encontremos el lugar buscado,
no lo copiamos (nos lo saltamos).
\trmblockidea

## Listas recursivas: buscar y descartar {.fragile}

\simpleTitle{3.- Haga una nueva lista, pero esquivando ``playa larga''}

\vspace{1ex}
\bld{Estrategia de resolución:}

- Si la lista está vacía:
    - Retorno una lista vacía.
- En caso contrario:
    - Obtengo la \nzinlinecode{cabeza} de la lista.
    - Si su atributo \ttt{lugar}  es "playa larga":
        - No hacemos nada con esta cabeza. 
        - Retorno la lista entregada por esta misma función, aplicada a la cola de la lista.
    - En caso contrario:
        - Creo una nueva lista con la misma cabeza y que por cola tenga a la aplicación de
        esta misma función, sobre la cola de la lista original.

## Listas recursivas: buscar y descartar {.fragile}

\simpleTitle{3.- Haga una nueva lista, pero esquivando ``playa larga''}

\footnotesize Se presenta una solución genérica, donde el segundo parámetro es el \ttt{lugar} a esquivar.

\begin{lstlisting}[style=frame02]
def esquivarLugar(vLista, buscado):
    if vLista == listaVacia:        # caso base
        return listaVacia
    else:                           # caso recursivo
        estaCarta = cabeza(vLista)
        resto = cola(vLista)        # obtengo ahora la cola
        if estaCarta.lugar == buscado:
            # no hago nada con esta cabeza: la esquivo
            return esquivarLugar(resto, buscado)
        else:
            # aplico esta búsqueda a la cola de la lista
            restoLimpio = esquivarLugar(resto, buscado)
            # retorno una nueva lista con TODO lo que está limpio
            return crearLista(estaCarta, restoLimpio)
\end{lstlisting}



## Listas recursivas: contar ocurrencias {.fragile}

\simpleTitle{4.- ¿Cuántos lugares distintos hay en toda la lista?}

\vspace{1ex}
\bld{Estrategia de resolución:}

- Si la lista está vacía:
    - Retorno cero.
- En caso contrario:
    - Obtengo la \nzinlinecode{cabeza} de la lista.
    - Obtengo el atributo \ttt{lugar} de la cabeza.
    - Obtengo la \nzinlinecode{cola} de la lista.
    - Utilizo la función \nzinlinecode{buscarLugar}, dándole como parámetros la cola y el atributo lugar de la cabeza.
    - Si ese lugar \bld{ya está también en la cola} (la función retorna True):
        - Retorno la aplicación recursiva de la función actual.
    - En caso contrario (no está en la cola):
        - Retorno 1 más la aplicación recursiva de la función actual.

## Listas recursivas: contar ocurrencias {.fragile}

\simpleTitle{4.- ¿Cuántos lugares distintos hay en toda la lista?}

\bgnblockidea
Para cada lugar encontrado, sólo se cuenta la última vez que se encuentra.
\trmblockidea

- Recorro la lista.
- Por cada ciudad que encuentre en cada paso recursivo:
    - Sólo la contaré como ciudad nueva cuando sepa que no la encontraré de nuevo más adentro en la lista.

## Listas recursivas: contar ocurrencias {.fragile}

\simpleTitle{4.- ¿Cuántos lugares distintos hay en toda la lista?}

\begin{lstlisting}[style=frame02]
def contarLugaresDistintos(vLista):
    if vLista == listaVacia:        # caso base
        return 0
    else:                           # caso recursivo
        estaCarta = cabeza(vLista)
        esteLugar = estaCarta.lugar
        resto = cola(vLista)
        if buscarLugar(resto, esteLugar):
            # dado que está repetida (ya está en algún lugar
            # de la cola), no la cuento
            return contarLugaresDistintos(resto)
        else:
            # como sé que no está más adentro, sí la cuento
            return 1 + contarLugaresDistintos(resto)
\end{lstlisting}


## Listas recursivas: buscar un máximo {.fragile}

\simpleTitle{5.- ¿Cuál es la profundidad más repetida en toda la lista?}

\vspace{1ex}
\bld{Estrategia de resolución:}

- En un principio, el máximo es 0.
    - Esto es una variable dentro de mi función.
- Obtengo la \nzinlinecode{cabeza} de la lista.
- Obtengo el atributo \ttt{profundidad} de la cabeza.
- Cuento cuántas veces se repite ese atributo dentro de toda la lista.
- Si esta cantidad es mayor que mi actual máximo, entonces ahora ella es
el nuevo máximo.
- Sigo revisando el resto de la lista.


## Listas recursivas: buscar un máximo {.fragile}

\simpleTitle{5.- ¿Cuál es la profundidad más repetida en toda la lista?}

- Pero... no sólo me basta con recordar la cantidad máxima de veces que
una profundidad está en la lista.

\pause

- También necesito \bld{saber el valor} de esa profundidad.

\pause

\bgnblockidea
Como necesito recordar dos valores: creo una nueva \bld{estructura} para almacenar
este par de valores.
\trmblockidea

\vspace{2ex}

\begin{lstlisting}[style=frame02]
estructura.crear("maximo", "profundidad total")
\end{lstlisting}

## Listas recursivas: buscar un máximo {.fragile}

\simpleTitle{5.- ¿Cuál es la profundidad más repetida en toda la lista?}

\begin{lstlisting}[style=frame02]
estructura.crear("maximo", "profundidad total")

def profundMasRepetida(vLista):
    # inicialmente, el máximo no se conoce
    actMaximo = maximo("", 0)
    # llamamos a una función que sí es recursiva
    actMaximo = profundMasRepetidaRecurs(vLista, actMaximo)
    return actMaximo.profundidad
\end{lstlisting}

## Listas recursivas: buscar un máximo {.fragile}

\simpleTitle{5.- ¿Cuál es la profundidad más repetida en toda la lista?}

\begin{lstlisting}[style=frame02]
def profundMasRepetidaRecurs(vLista, actMaximo):
    if vLista == listaVacia:        # caso base
        return actMaximo
    else:                           # caso recursivo
        estaProfundidad = cabeza(vLista).profundidad
        nVeces = contarProfundidad(vLista, estaProfundidad)
        if nVeces > actMaximo.total:
            actMaximo = maximo(estaProfundidad, nVeces)
        return profundMasRepetidaRecurs(cola(vLista), actMaximo)
\end{lstlisting}

## Listas recursivas: preguntas y propósito {.fragile}

\simpleTitle{Preguntas útiles de responder:}

1. ¿Hay alguna carta que nos lleve a la ``isla gaviota''?
    - \structure{Búsqueda de un elemento en una lista.}
\pause
1. ¿Cuántas cartas nos harán excavar 4 metros de profundidad?
    - \structure{Contar elementos que cumplan una condición.}
\pause
1. Haga una nueva lista, pero esquivando ``playa larga''.
    - \structure{Duplicar una lista, aplicando condiciones.}
\pause
1. ¿Cuántos lugares distintos hay en toda la lista?
    - \structure{Búsqueda condicionada.}
\pause
1. ¿Cuál es la profundidad más repetida en toda la lista?
    - \structure{Búsqueda de máximos y/o mínimos.}

## Listas recursivas: problemas propuestos {.fragile}

- En DocenciaWeb está disponible un módulo llamado \structure{\ttt{cartas.py}}.
    - Provee una definición de la estructura \ttt{carta} más \itt{ad-hoc} a las ciudades.
    - Provee dos funciones:
        1. \ttt{crear15Cartas()}: entrega una lista con 15 cartas.
        1. \ttt{crearMuchasCartas()}: entrega una lista con muchas cartas.

- Usando estas herramientas, programe lo siguiente:
    1. Una función que reciba una lista de cartas y retorne el
    listado de todas las ciudades involucradas, sin repetición.
    1. Una función que reciba una lista de cartas y retorne el
    listado de todas las ciudades involucradas, sin repetición, indicando
    para cada ciudad, cuántas veces aparece en la lista.
