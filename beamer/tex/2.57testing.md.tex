## Repaso: recursividad y estimación de PI {.fragile}

\bgnenviron{<1-2>}{onlyenv}

- Todos sabemos que la constante $\pi$ es un número irracional, y por lo tanto, infinito.

\begin{center}
\textsf{$\pi$ = 3,14159265358979323846264338327950288419716939$\ldots$}
\end{center}

\trmenviron{onlyenv}

\bgnenviron{<2-3>}{onlyenv}

- Pero existe la llamada ``serie de Leibniz'' con la cual se puede calcular un valor aproximado de $\pi$:

\vspace{-3ex}
$$ \frac{\pi}{4} = \left( 1 - \frac{1}{3} + \frac{1}{5} - \frac{1}{7} + \frac{1}{9} - \frac{1}{11} + \ldots + \frac{(-1)^{n}}{2n + 1} \right) = \sum_{n=0}^{N} \frac{(-1)^{n}}{2n + 1} $$

\trmenviron{onlyenv}

\bgnenviron{<2>}{onlyenv}

- Mientras más grande sea $N$, más cerca estaremos del valor ``verdadero'' de $\pi$.

\trmenviron{onlyenv}

\bgnenviron{<3-6>}{onlyenv}

\simpleTitle{Solución recursiva:}

- \bld{Caso base:}
    - $n = 0$
    - Resultado: $1$
- \bld{Caso recursivo:} Retorno la suma de:
    - El n-ésimo valor de la serie.\tikzmark{markNthValueText}
    - La función reducida a $n \tikzmark{markRecCallText}- 1$.

\trmenviron{onlyenv}

\bgnenviron{<5-6>}{onlyenv}

\vspace{2ex}

\begin{lstlisting}[style=frame02]
def calcPI(n):
    return 4 * seriePI4(n)

def seriePI4(n):
    if n == 0:
        return 1
    else:
        return (*@\tikzmark{markBgnNthValue}@*)(-1)**(n) / (2*n + 1.0)(*@\tikzmark{markTrmNthValue}@*) + (*@\tikzmark{markBgnRecCall}@*)seriePI4(n - 1)(*@\tikzmark{markTrmRecCall}@*)
\end{lstlisting}

\trmenviron{onlyenv}

\bgnenviron{<6>}{onlyenv}

\braceUpwards[decoration/.style={decoration={brace,raise=0pt,amplitude=0.5em},decorate,ultra thick,structure!80,font=\footnotesize}]{markBgnNthValue}{markTrmNthValue}
\braceUpwards[decoration/.style={decoration={brace,raise=0pt,amplitude=0.5em},decorate,ultra thick,structure!80,font=\footnotesize}]{markBgnRecCall}{markTrmRecCall}

\begin{tikzpicture}[remember picture, overlay]
    \draw[thick, structure!80,->,-Stealth] ([shift={(0,.5ex)}]pic cs:markNthValueText) to[out=0,in=90] ($(markBgnNthValuemarkTrmNthValue)+(0,5mm)$);
    \draw[thick, structure!80,->,-Stealth] (pic cs:markRecCallText) to[out=-90,in=90] ($(markBgnRecCallmarkTrmRecCall)+(0,5mm)$);
\end{tikzpicture}

\trmenviron{onlyenv}

## Repaso: recursividad y estimación de PI {.fragile}

\simpleTitle{Observaciones:}

- Esta serie es conocida por ``converger'' muy lentamente.
    - Hay que entregar un $n$ muy alto para que empiecen a aparecer los primeros decimales conocidos.


\bgnblockgood
\bld{Ejercicio propuesto:} Implementa la ``Transformación de Shanks de la Serie de Leibniz'', que es
una serie que converge más rápido:

\vspace{-3ex}
$$ \frac{\pi}{4} = \sum_{i=0}^{N} \left( \frac{1}{4n + 1} - \frac{1}{4n + 3} \right) $$

\trmblockgood

## Testing más en profundidad {.fragile}

- Hasta ahora hemos usado \nzinlinecode{assert} con cálculos exactos.

\pause

\bgnblockdanger
¡No siempre se puede hacer eso!
\trmblockdanger

1. Hay veces que tendremos un resultado del cual sólo sabemos que está en un \structure{rango determinado}.
1. Otras veces el \structure{cálculo} es \structure{aproximado}, y nunca tendremos un valor exacto.

## Testing: rangos {.fragile}

\simpleTitle{Cálculos en un rango determinado}

- El típico ejemplo es el de los números aleatorios.

- Generemos números aleatorios, en el rango [0, 1):

\begin{lstlisting}
>>> import random
>>>
>>> aleat = random.random()
>>> print(aleat)
0.187325912854
\end{lstlisting}

\bgnblockgood
¿Y si queremos simular el lanzamiento de un dado?

Eso equivale a
generar números aleatorios enteros en el rango [1, 6].
\trmblockgood

## Testing: rangos {.fragile}

\begin{lstlisting}[style=frame02]
# tirarDado: None -> int
# Retorna un entero aleatorio en el rango [1, 6])
def tirarDado():
    return int(6 * random.random()) + 1
# Tests
# ????
\end{lstlisting}

\pause

\bgnblockdanger
¿Pero qué hacemos con los tests?
\trmblockdanger

\pause

\begin{lstlisting}[style=frame02]
# tirarDado: None -> int
# Retorna un entero aleatorio en el rango [1, 6])
def tirarDado():
    return int(6 * random.random()) + 1
# Tests
assert 1 <= tirarDado() <= 6
\end{lstlisting}

## Testing: cálculos aproximados {.fragile}

\simpleTitle{Cálculos aproximados}

- Uno de los casos más típicos corresponde a los cálculos donde están involucrados números reales $(\mathbb{R})$.
    - Peor aún si se trata de irracionales $(\mathbb{I})$.

\begin{lstlisting}
>>> print(1.0/2.0)
0.5
>>> print(0.5 == 1.0/2.0)
True
>>> print(1.0/3.0)
0.333333333333(*@\tikzmark{markSameNumber01}@*)
>>> print(0.33333333333(*@\tikzmark{markSameNumber02}@*)3 == 1.0/3.0)
False
\end{lstlisting}

\pause

\bgnblockidea
Por lo general es una mala idea comparar igualdad entre números reales.
\trmblockidea

\drawTikZComment[above right=6ex and 16ex of pic cs:markSameNumber02]{markSameNumber02}{¡Parecen ser el mismo número!}
\drawTikZAnotherArrow{markSameNumber01}{markSameNumber02-comment}

## Testing: cálculos aproximados {.fragile}

\simpleTitle{Cálculos aproximados}

- Veamos un ejemplo aún más extraño:

\begin{lstlisting}
>>> print(0.3 == 0.1 + 0.2)
False(*@\tikzmark{markThreeTwoOne}@*)
\end{lstlisting}

\drawTikZComment[right=22ex of pic cs:markThreeTwoOne]{markThreeTwoOne}{¿Por qué?}
\vspace{-2ex}

\pause

\begin{lstlisting}
>>> print(0.3 - (0.1 + 0.2))
-5.551115123125783e-17
\end{lstlisting}

- Veamos esos tres números ``más en detalle'':

\begin{lstlisting}
>>> # Imprimamos 0.1, 0.2 y 0.3 con 18 decimales:
>>> print("{0:.18f}".format(0.1))
0.100000000000000006(*@\tikzmark{markBadDecimals01}@*)
>>> print("{0:.18f}".format(0.2))
0.200000000000000011(*@\tikzmark{markBadDecimals02}@*)
>>> print("{0:.18f}".format(0.3))
0.299999999999999989(*@\tikzmark{markBadDecimals03}@*)
\end{lstlisting}

\drawTikZComment[right=22ex of pic cs:markBadDecimals02]{markBadDecimals02}{Esos últimos decimales son la causa}
\drawTikZAnotherArrow{markBadDecimals01}{markBadDecimals02-comment}
\drawTikZAnotherArrow{markBadDecimals03}{markBadDecimals02-comment}


## Testing: cálculos aproximados {.fragile}

\simpleTitle{Cálculos aproximados}

- Ante este problema, mejor adaptémonos.
- En vez de buscar la máxima exactitud, la mayoría de las veces sólo es necesario un \structure{margen
de error aceptable}.

\begin{lstlisting}
>>> print(0.3 == 1.0 + 2.0)
False

>>> print(0.3 - (0.1 + 0.2))
-5.551115123125783e-17

>>> print(abs(0.3 - (0.1 + 0.2)) < 0.001)
True
\end{lstlisting}
