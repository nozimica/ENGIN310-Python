## Testing más en profundidad {.fragile}

- Hasta ahora hemos usado \nzinlinecode{assert} con cálculos exactos.

\pause

\bgnblockdanger
¡No siempre se puede hacer eso!
\trmblockdanger

1. Hay veces que tendremos un resultado del cual sólo sabemos que está en un \structure{rango determinado}.
1. Otras veces el \structure{cálculo} es \structure{aproximado}, y nunca tendremos un valor exacto.

## Testing: rangos {.fragile}

\simpleTitle{Cálculos en un rango determinado}

- El típico ejemplo es el de los números aleatorios.

- Generemos números aleatorios, en el rango [0, 1):

\begin{lstlisting}
>>> import random
>>>
>>> aleat = random.random()
>>> print(aleat)
0.187325912854
\end{lstlisting}

\bgnblockgood
¿Y si queremos simular el lanzamiento de un dado?

Eso equivale a
generar números aleatorios enteros en el rango [1, 6].
\trmblockgood

## Testing: rangos {.fragile}

\begin{lstlisting}[style=frame02]
# tirarDado: None -> int
# Retorna un entero aleatorio en el rango [1, 6]
def tirarDado():
    return int(6 * random.random()) + 1
# Tests
# ????
\end{lstlisting}

\pause

\bgnblockdanger
¿Pero qué hacemos con los tests?
\trmblockdanger

\pause

\begin{lstlisting}[style=frame02]
# tirarDado: None -> int
# Retorna un entero aleatorio en el rango [1, 6]
def tirarDado():
    return int(6 * random.random()) + 1
# Tests
assert 1 <= tirarDado() <= 6
\end{lstlisting}

## Testing: cálculos aproximados {.fragile}

- Uno de los casos más típicos corresponde a los cálculos donde están involucrados números reales $(\mathbb{R})$.
    - Peor aún si se trata de irracionales $(\mathbb{I})$.

\begin{lstlisting}
>>> print(1.0/2.0)
0.5
>>> print(0.5 == 1.0/2.0)
True
>>> print(1.0/3.0)
0.333333333333(*@\tikzmark{markSameNumber01}@*)
>>> print(0.33333333333(*@\tikzmark{markSameNumber02}@*)3 == 1.0/3.0)
False
\end{lstlisting}

\pause

\bgnblockidea
Por lo general es una mala idea comparar igualdad entre números reales.
\trmblockidea

\drawTikZComment[pos={above right},len={6ex and 16ex}]{markSameNumber02}{¡Parecen ser el mismo número!}
\drawTikZAnotherArrow{markSameNumber01}{markSameNumber02-comment}

## Testing: cálculos aproximados {.fragile}

- Veamos un ejemplo aún más extraño:

\begin{lstlisting}
>>> print(0.3 == 0.1 + 0.2)
False(*@\tikzmark{markThreeTwoOne}@*)
\end{lstlisting}

\drawTikZComment[pos={right},len={22ex}]{markThreeTwoOne}{¿Por qué?}

\pause

\vspace{-2ex}
\begin{lstlisting}
>>> print(0.3 - (0.1 + 0.2))
-5.551115123125783e-17
\end{lstlisting}

- Veamos esos tres números ``más en detalle'':

\begin{lstlisting}
>>> # Imprimamos 0.1, 0.2 y 0.3 con 18 decimales:
>>> print("{0:.18f}".format(0.1))
0.100000000000000006(*@\tikzmark{markBadDecimals01}@*)
>>> print("{0:.18f}".format(0.2))
0.200000000000000011(*@\tikzmark{markBadDecimals02}@*)
>>> print("{0:.18f}".format(0.3))
0.299999999999999989(*@\tikzmark{markBadDecimals03}@*)
\end{lstlisting}

\drawTikZComment[pos={right},len={22ex}]{markBadDecimals02}{Esos últimos decimales son la causa}
\drawTikZAnotherArrow{markBadDecimals01}{markBadDecimals02-comment}
\drawTikZAnotherArrow{markBadDecimals03}{markBadDecimals02-comment}


## Testing: cálculos aproximados {.fragile}

- Ante este problema, mejor adaptémonos.

\bgnblockidea
En vez de buscar la máxima exactitud, la mayoría de las veces sólo es necesario un \structure{margen
de\tikzmark{markEpsilon01} error aceptable}.
\trmblockidea

\begin{lstlisting}
>>> print(0.3 == 1.0 + 2.0)
False

>>> print(0.3 - (0.1 + 0.2))
-5.551115123125783e-17

>>> print(abs(0.3 - (0.1 + 0.2)) <(*@\tikzmark{markEpsilon02}@*) 0.001)
True
\end{lstlisting}

\begin{tikzpicture}[remember picture, overlay]
    \draw[thick, structure!80,->,-Stealth] (pic cs:markEpsilon01) to[out=-90,in=90] ($(pic cs:markEpsilon02)+(0,5mm)$);
\end{tikzpicture}

## Testing: cálculos aproximados {.fragile}

- Esto es un cálculo que tendremos que hacer muy seguido.

\bgnblockidea
Por ello, definiremos una nueva función auxiliar: \nzinlinecode{cerca()}.
\trmblockidea

\vfill

\begin{lstlisting}[style=frame02]
# cerca: num num num -> bool
# Retorna True si la diferencia entre x e y es menor a epsilon.
# Ejemplo: cerca(0.33333, 1/3.0, 0.001) retorna True
def cerca(x, y, epsilon):
    diff = x - y
    return abs(diff) < epsilon
\end{lstlisting}

## Testing: cálculo de la raíz cuadrada {.fragile}

\bgnblockgood
\strongText{Ejemplo 1:} Calcularemos raíces cuadradas, usando el\newline ``método de Herón''
\trmblockgood

- Herón de Alejandría (ingeniero y matemático helenístico del siglo I).
- Entre otras muchas cosas, propuso un método recursivo para calcular la aproximación de la raíz cuadrada de un número.

\pause

$$ \sqrt{x} = \lim_{n \to \infty} r_{n+1} \qquad \text{tal que:}  \begin{cases} r_{n+1} & = \frac{1}{2} \left(r_n + \frac{x}{r_n}\right) \\ r_0 & = \text{\footnotesize{número arbitrario positivo}} \\ \end{cases} $$

## Testing: cálculo de la raíz cuadrada {.fragile}

\vspace{-3ex}
$$ \sqrt{x} = \lim_{n \to \infty} r_{n+1} \qquad \text{tal que:}  \begin{cases} r_{n+1} & = \frac{1}{2} \left(r_n + \frac{x}{r_n}\right) \\ r_0 & = \text{\footnotesize{número arbitrario positivo}} \\ \end{cases} $$

- Usemos esta definición para calcular $\sqrt{11}$:

\vspace{-1ex}
\begin{small}
\begin{align*}
    \onslide<2->{r_0 &= 11 / 2 = 5.5 \\}
    \onslide<3->{r_1 &= \frac{1}{2} \left(r_0 + \frac{11}{r_0}\right) = \frac{1}{2} \left(5,5 + \frac{11}{5,5}\right)                           = \frac{1}{2} \left(7.5\right)             = 3.75 \\}
    \onslide<4->{r_2 &= \frac{1}{2} \left(r_1 + \frac{11}{r_1}\right) = \frac{1}{2} \left(3.75 + \frac{11}{3.75}\right)                         = \frac{1}{2} \left(6.68\overline{3}\right) = 3.341\overline{6} \\}
    \onslide<5->{r_3 &= \frac{1}{2} \left(r_2 + \frac{11}{r_2}\right) = \frac{1}{2} \left(3.341\overline{6} + \frac{11}{3.341\overline{6}}\right) = \frac{1}{2} \left(6.6334\right)         = 3.3167 \\}
    \onslide<6->{r_4 &= \frac{1}{2} \left(r_3 + \frac{11}{r_3}\right) = \frac{1}{2} \left(3.3167 + \frac{11}{3.3167}\right)                     = \frac{1}{2} \left(6.6333\right)          = 3.3166 \\}
\end{align*}
\end{small}

## Testing: cálculo de la raíz cuadrada {.fragile}

- Por la teoría, sabemos que este cálculo funciona.
    - Recién lo probamos una vez.
- ¿Cómo podríamos programar este cálculo de manera recursiva?

\pause

\bgnblockidea
\raggedright
La clave es modificar nuestra estrategia: \newline\newline
\bld{Caso base:} al llegar a un valor razonablemente cercano.\newline
\bld{Caso recursivo:} estimar $r_{n+1}$ en función de $r_n$.
\trmblockidea

## Testing: cálculo de la raíz cuadrada {.fragile}

Resumiendo:

- \bld{Caso base:} al llegar a un valor razonablemente cercano.
    - Equivale a terminar la recursión cuando el cuadrado de nuestra estimación esté a una distancia $\epsilon$ de $x$.
- \bld{Caso recursivo:} estimar $r_{n+1}$ en función de $r_n$.

\pause

\begin{lstlisting}[style=frame02]
# heronRaizRec: num num num -> num
# Calcula la raíz cuadrada de x con una precisión epsilon,
# y un valor inicial de la estimación.
# Ejemplo: heronRaizRec(13, 0.001, 13/2.0) retorna 3.6055512902583184
def heronRaizRec(x, epsilon, estimacion):
    if cerca(estimacion**2, x, epsilon):  # caso base
        return estimacion
    else:                                 # caso recursivo
        nuevaEstimacion = (estimacion + (x / estimacion)) / 2
        return heronRaizRec(x, epsilon, nuevaEstimacion)
\end{lstlisting}

## Testing: cálculo de la raíz cuadrada {.fragile}

- Hay algo en la función anterior que nos debería molestar: \pause
- Para calcular una raíz cuadrada, hay que entregarle una estimación inicial.
    - Muy pronto se convertirá en algo incómodo.

\pause

\bgnblockidea
\raggedright

1. Dejamos a la función recursiva como una \structure{función auxiliar}.
1. Definimos una función que sólo le pide al usuario lo indispensable.
1. Esta función inicia la recursividad, con los valores iniciales correctos.

\trmblockidea

## Testing: cálculo de la raíz cuadrada {.fragile}

\begin{lstlisting}[style=frame03]
# heronRaizRec: num num num -> num
# Calcula la raíz cuadrada de x con una precisión epsilon,
# y un valor inicial de la estimación.
# Ejemplo: heronRaizRec(13, 0.001, 13/2.0) retorna 3.6055512902583184
def heronRaizRec(x, epsilon, estimacion):
    if cerca(estimacion**2, x, epsilon):  # caso base
        return estimacion
    else:                                 # caso recursivo
        nuevaEstimacion = (estimacion + (x / estimacion)) / 2
        return heronRaizRec(x, epsilon, nuevaEstimacion)

# heronRaiz: num num -> num
# Calcula la raíz cuadrada de x con una precisión epsilon.
# Ejemplo: heronRaiz(13, 0.001) retorna 3.6055512902583184
def heronRaiz(x, epsilon):
    return heronRaizRec(x, epsilon, x/2.0)

# Probamos la función con la verdadera sqrt
import math
assert cerca(heronRaiz(3, 0.001), math.sqrt(3), 0.001)
assert cerca(heronRaiz(5, 0.001), math.sqrt(5), 0.001)
assert cerca(heronRaiz(11, 0.001), math.sqrt(11), 0.001)
\end{lstlisting}

