## Ejemplo: jugar al gato {.fragile}

- Ahora veremos cómo implementar un programa que le permita a dos personas ``\bld{jugar al gato}''.
\pause

- Para modelar bien este problema, debemos dividirlo en subproblemas más simples:

\bgnblockgood

1. Representar el tablero de juego.
1. Imprimir en pantalla este tablero.
1. Modelar el ingreso de las jugadas.
1. Determinar al ganador o declarar el juego como empatado.

\trmblockgood

## Ejemplo: jugar al gato {.fragile}

\simpleTitle{Representar el tablero:}

\bgncolumns
\column{0.45\textwidth}
Como sabemos, el tablero es una matriz de 3x3:

\bgnblockgood
    \centering
    \tictactoe{}{}{}{}{}{}{}{}{}
\trmblockgood

\column{0.1\textwidth}

\column{0.45\textwidth}
Donde podemos darle un índice a cada celda:

\bgnblockgood
    \centering
    \tictactoe{1}{2}{3}{4}{5}{6}{7}{8}{9}
\trmblockgood

\trmcolumns

\pause

\vfill

Pero en computación, muchas veces es más fácil representar las matrices como un solo
gran vector, poniendo cada fila después de la anterior:

\bgnblocknormal[centered,wd=70mm]
    \centering
    \tictactoelinear{1}{2}{3}{4}{5}{6}{7}{8}{9}
\trmblocknormal

\vfill

Precisamente eso haremos nosotros en esta implementación.

## Ejemplo: jugar al gato

\simpleTitle{Representar el tablero}

- Ahora bien, también sabemos que cada casilla del tablero puede tener tres estados:

\bgnblockgood

1. Vacía.
1. Marcada como jugada por el primer jugador.
1. Marcada como jugada por el segundo jugador.

\trmblockgood

- ¿Cómo los podemos representar?

## Ejemplo: jugar al gato

\simpleTitle{Representar el tablero}

- El tablero de juego puede representarse dentro del programa como un \bld{vector de enteros}, donde cada uno de los tres
estados posibles es representado por los valores \btt{0}, \btt{1} y \btt{-1}:

\bgnblockgood

\begin{description}
    \item[0:] Vacía.
    \item[1:] Marcada como jugada por el primer jugador.
    \item[-1:] Marcada como jugada por el segundo jugador.
\end{description}

\trmblockgood


## Ejemplo: jugar al gato

\simpleTitle{Imprimir el tablero en consola.}

- Recordemos que representamos el tablero como un solo gran
vector, con cada una de las tres filas una al lado de la otra:

\bgnblockgood[centered,wd=70mm]
    \centering
    \tictactoelinear{1}{2}{3}{4}{5}{6}{7}{8}{9}
\trmblockgood

\pause

- Pero para imprimirlo en pantalla hay que desplegarlo en su forma de tres filas por tres columnas.

\pause

- Qué hacer: recorremos el vector, imprimimos el valor actual de cada celda, pero cuando sea el momento de ``pasar'' a
una nueva fila, simplemente imprimimos un salto de línea.
    - Debemos saber identificar cuándo hemos pasado a una nueva fila.

## Ejemplo: jugar al gato

\simpleTitle{Representar el tablero}

- Hemos llegado a una nueva fila simplemente cuando ya hayamos desplegado tres valores en pantalla:

\begin{tabular}{c}
    \parbox{\textwidth}{%
    \bgnblockgood[centered,wd=70mm]
        \centering
        \tictactoelinear{1}{2}{3}{\textcolor{blue}{4}}{5}{6}{\textcolor{blue}{7}}{8}{9}
    \trmblockgood
    } \\
    $\Bigg\downarrow$ \\
    \parbox{\textwidth}{%
    \bgnblockgood[centered,wd=40mm]
        \centering
        \tictactoe{1}{2}{3}{4}{5}{6}{7}{8}{9}
    \trmblockgood
    }
\end{tabular}

## Ejemplo: jugar al gato {.fragile}

\simpleTitle{Modelar el ingreso de jugadas:}


- Una jugada en este juego se representa como la celda
que un jugador quiere ``marcar para sí mismo'':

\vfill

\begin{center}
\begin{tabular}{ccc}
  \parbox{24mm}{%
    \bgnblockgood[centered,wd=22mm]
    \centering
    \tictactoe{}{}{}{}{}{X}{}{}{}
    \trmblockgood
  }
  & $\xrightarrow{\quad jugada\quad}$ &
  \parbox{24mm}{%
    \bgnblockgood[centered,wd=22mm]
    \centering
    \tictactoe{O}{}{}{}{}{X}{}{}{}
    \trmblockgood
  }
\end{tabular}
\end{center}

\pause
\vfill

- Por lo tanto, para cada uno de los dos jugadores, debemos en cada turno:
    - Preguntarles la jugada que quieren hacer.
    - Procesar su respuesta.
    - Actualizar adecuadamente el tablero de juego.


## Ejemplo: jugar al gato {.fragile}

\simpleTitle{Modelar el ingreso de jugadas:}

- Una manera simple de hacer esto es sólo pedirle al jugador el índice
de la celda que quiere marcar.

- Este índice corresponde a un índice del vector con el cual
representamos el tablero dentro del programa.


## Ejemplo: jugar al gato {.fragile}

\simpleTitle{Determinar el resultado:}

- La condición de que un jugador \bld{ganó el juego} es simple:
    - Marcó las tres casillas de una misma fila.
    - Marcó las tres casillas de una misma columna.
    - Marcó las tres casillas de una de las dos grandes diagonales.

\def\XX{\textbf{\textcolor{blue}{X}}}

\vspace{-1ex}
\begin{center}
\begin{tabular}{ccccc}
  \parbox{24mm}{%
    %\vspace{-10mm}\hspace*{-6mm}
    \bgnblockgood[centered,wd=22mm]
    \centering\hspace{0mm}
    \tictactoe{\XX}{\XX}{\XX}{O}{X}{O}{O}{}{O}
    \trmblockgood
  }
  & &
  \parbox{24mm}{%
    %\vspace{-10mm}\hspace*{-6mm}
    \bgnblockgood[centered,wd=22mm]
    \centering\hspace{0mm}
    \tictactoe{X}{\XX}{O}{O}{\XX}{O}{}{\XX}{}
    \trmblockgood
  }
  & &
  \parbox{24mm}{%
    %\vspace{-10mm}\hspace*{-6mm}
    \bgnblockgood[centered,wd=22mm]
    \centering\hspace{0mm}
    \tictactoe{\XX}{O}{X}{}{\XX}{}{O}{O}{\XX}
    \trmblockgood
  }
\end{tabular}
\end{center}

## Ejemplo: jugar al gato {.fragile}

\simpleTitle{Determinar el resultado:}

- Por lo tanto, luego de cada jugada se debe recorrer el vector que representa al tablero:
    - Para cada una de sus tres filas, sus tres columnas y sus dos diagonales:
        - Buscar si están todas sus tres celdas marcadas por un mismo jugador.

## Ejemplo: jugar al gato {.fragile}

\simpleTitle{Determinar el resultado:}

- Para saber si \bld{hubo empate}, habría que detectar cuando no hay posibilidad de que ninguno de los jugadores
gane, lo que implica: 
    - Las tres filas tienen celdas marcadas para ambos jugadores.
    - Las tres columnas tienen celdas marcadas para ambos jugadores.
    - Las dos diagonales tienen celdas marcadas para ambos jugadores.


## Ejemplo: jugar al gato {.fragile}

\simpleTitle{Determinar el resultado:}

- Sin embargo en una primera aproximación simplemente contaremos la cantidad total de jugadas, y si llegamos al
máximo de nueve posibles, declararemos el juego como empatado.

\bgnblockdanger
¿Qué consecuencia molesta para los jugadores implicará esta decisión?
\trmblockdanger

## Ejemplo: jugar al gato {.fragile}

\simpleTitle{Implementación en Python:}

- Para implementar este juego, debemos contar con las siguientes variables:
    - Un vector de nueve enteros para representar el tablero.
    - Una variable que nos indique cuál jugador tiene el turno de juego.
    - Un contador para saber si hemos llegado al final del juego (empate).


- Y debemos hacer un ciclo por cada turno de juego, dentro del cual:
    - Mostramos el estado actual del tablero de juego (función aparte).
    - Le pedimos su jugada a quien tiene el turno actual.
    - Marcamos en el tablero la jugada.
    - Verificamos si el jugador actual ganó (función aparte).

