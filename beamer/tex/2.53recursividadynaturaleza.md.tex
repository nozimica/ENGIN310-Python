## Recursividad: Conejitos cariñosos {.fragile}

\setbeamertemplate{itemize/enumerate body begin}{\footnotesize}
\setbeamertemplate{itemize/enumerate subbody begin}{\footnotesize}

\vspace{-2ex}

- Mes 0: De una casa se escapa una pareja de conejos bebés.
- Mes 1: Estos conejos, al mes ya son adultos.
- Mes 2: Ya tienen un par de crías.
    - Son de una raza que siempre paren un macho y una hembra.
- Mes 3: La pareja original tiene otro par de crías.
    - La primera pareja de hijos ya es adulta.
- Mes 4: La pareja original de nuevo tiene otro par de crías.
    - La primera pareja de hijos también aporta con un par de crías.
    - La segunda pareja de hijos ya es adulta.
- Mes 5: La pareja original de nuevo tiene otro par de crías.
    - La primera pareja de hijos tiene su segundo par de crías.
    - La segunda pareja de hijos pare por primera vez.
    - La tercera pareja de hijos ya es adulta.
- Así, sucesivamente...

\bgnblocknormal
¿Cuántas parejas de conejos hay en el mes \bld{n}?
\trmblocknormal

\placeLogo{img/rabbit-1f407.pdf}

## Recursividad: Conejitos cariñosos {.fragile}

\newcommand{\ing}[1]{\includegraphics[width=12mm]{#1}}
\newcommand{\nzRbb}{\ing{img/twoRabbits.pdf}}
\tikzset{
  % style for inserting images as nodes
  img/.style={
    % text width=8mm,
    minimum height=4mm,
    anchor=center,
    %text height=2cm,                 %% don't use this
    inner sep=0pt,     %% use this
    outer sep=0pt,     %% and this
    % rectangle,
    % align=center,
    % draw,thick % only for debugging..
  },
  treeLines/.style={
    ultra thick, black
  },
}
\begin{tikzpicture}

  \matrix (rbbs) [matrix of nodes, nodes=img, row sep=4mm, column sep=0mm]{
   \footnotesize{Mes}  &        &        &        &        &        &        &        &        & \footnotesize{Parejas} \\
   0 &        &        &        &        & \nzRbb &        &        &        & 1       \\
   1 &        &        &        &        & \nzRbb &        &        &        & 1       \\
   2 &        & \nzRbb &        &        & \nzRbb &        &        &        & 2      \\
   3 &        & \nzRbb &        &        & \nzRbb &        & \nzRbb &        & 3      \\
   4 & \nzRbb & \nzRbb &        & \nzRbb & \nzRbb &        & \nzRbb &        & 5      \\
   5 & \nzRbb & \nzRbb & \nzRbb & \nzRbb & \nzRbb & \nzRbb & \nzRbb & \nzRbb & 8      \\
  };

% parejas
\draw[treeLines] (rbbs-2-6) -- (rbbs-3-6);
\draw[treeLines] (rbbs-3-6) -- (rbbs-4-6);
\draw[treeLines] (rbbs-4-6) -- (rbbs-5-6);
\draw[treeLines] (rbbs-5-6) -- (rbbs-6-6);
\draw[treeLines] (rbbs-6-6) -- (rbbs-7-6);
% parejas
\draw[treeLines] (rbbs-4-3) -- (rbbs-5-3);
\draw[treeLines] (rbbs-5-3) -- (rbbs-6-3);
\draw[treeLines] (rbbs-6-3) -- (rbbs-7-3);
% parejas
\draw[treeLines] (rbbs-5-8) -- (rbbs-6-8);
\draw[treeLines] (rbbs-6-8) -- (rbbs-7-8);
% parejas
\draw[treeLines] (rbbs-6-2) -- (rbbs-7-2);
% parejas
\draw[treeLines] (rbbs-6-5) -- (rbbs-7-5);

% hijos
\draw[treeLines] (rbbs-3-6) -- (rbbs-4-3);
\draw[treeLines] (rbbs-4-6) -- (rbbs-5-8);
\draw[treeLines] (rbbs-5-3) -- (rbbs-6-2);
\draw[treeLines] (rbbs-5-6) -- (rbbs-6-5);
\draw[treeLines] (rbbs-6-3) -- (rbbs-7-4);
\draw[treeLines] (rbbs-6-6) -- (rbbs-7-7);
\draw[treeLines] (rbbs-6-8) -- (rbbs-7-9);

\end{tikzpicture}

## Recursividad: Sucesión de Fibonacci {.fragile}

\bgnblocknormal
\bld{La sucesión de Fibonacci} se define recursivamente como:
\trmblocknormal

\vspace{-2ex}

$$ F_n = \begin{cases}
        n                 & \;\;\;\;\text{si} \;\;\;\; 0 \leq n \leq 1 \\
        F_{n-1} + F_{n-2} & \;\;\;\;\text{si} \;\;\;\; n > 1 \\
    \end{cases}
$$

\vspace{1ex}
\begin{lstlisting}[style=frame02]
# fibonacci: int -> int
# Calcula el n-ésimo valor de la sucesión de Fibonacci.
# Ejemplo: fibonacci(8) retorna 21
def fibonacci(n):
    if n <= 1:            # caso base
        return n
    else:                 # caso recursivo
        return fibonacci(n - 1) + fibonacci(n - 2)
# Tests
assert fibonacci(0) == 0
assert fibonacci(9) == 34
\end{lstlisting}

