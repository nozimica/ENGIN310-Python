## Tipos de dato lógico

- Hasta ahora hemos visto variables numéricas, y en menor medida, de texto.

- He aquí un nuevo tipo de dato, muy sencillo:

\pause

\bgnblockdefinition
\bld{Tipo de dato lógico:} corresponden a la respuesta a preguntas lógicas, donde las
únicas opciones son\newline \structure{True} o \structure{False}.
\trmblockdefinition

\vspace{.6ex}

\scalebox{.7}{%
\begin{tikzpicture}[every node/.style={cloud callout, draw, fill=Wheat1!50, cloud ignores aspect, cloud puffs=12, text width=5em, text centered}]
\node at (0,0) {\bf ¿Es $n$ positivo?};
\node at (4,1) {\bf ¿Está el producto en oferta?};
\node at (8,-1) {\bf ¿Es el vuelto > 0?};
\node at (12,0) {\bf ¿Quedan elementos en la lista?};
\end{tikzpicture}
}

## Tipos de dato lógico: {.fragile}

\simpleTitle{Los operadores condicionales más importantes son:}

\begin{zebraTable}[0.5]%
    \rowcolor{structure}%
    \textcolor{white}{Operador} & \textcolor{white}{Significado} \\
    \nzinlinecode{==} &  Igualdad \\
    \nzinlinecode{!=} & Diferentes \\
    \nzinlinecode{not} &  Negación \\
    \nzinlinecode{<} &  Menor que \\
    \nzinlinecode{>} &  Mayor que \\
    \nzinlinecode{<=} & Menor o igual que \\
    \nzinlinecode{>=} & Mayor o igual que \\
    \nzinlinecode{and} &  Y \\
    \nzinlinecode{or} &   O \\
\end{zebraTable}


## Tipos de dato lógico: {.fragile}

\simpleTitle{Probemos algunas expresiones en Python:}

\vspace{-3ex}

\bgncolumns

\column{0.35\textwidth}

\begin{lstlisting}
>>> 7 < 10
True

>>> 8 < 2
False

>>> 8 > 4
True

>>> 8 > 11
False

>>> 4 > 4
False
\end{lstlisting}

\column{0.35\textwidth}

\begin{lstlisting}
>>> 3 == 6
False

>>> 6 == 3 * 2
True

>>> 10 != 7
True

>>> 8 <= 10
True

>>> 10 >= 3**2
True
\end{lstlisting}

\column{0.35\textwidth}

\begin{lstlisting}
>>> True == True
True

>>> False == True
False

>>> False == False
True
\end{lstlisting}

\trmcolumns


## Tipos de dato lógico: {.fragile}

\simpleTitle{Probemos algunas expresiones con variables:}

\vspace{-3ex}

\bgncolumns

\column{0.35\textwidth}

\begin{lstlisting}
>>> largo = 12
>>> ancho = 7
>>> alto = 6
>>> listo = True
>>> suficiente = 90

>>> largo > ancho
True

>>> largo < 2 * ancho
True

>>> alto == largo / 2
True
\end{lstlisting}

\column{0.65\textwidth}

\begin{lstlisting}
>>> alto <= largo
True

>>> alto * ancho < largo
False

>>> listo (*@\tikzmark{bgnSetbool}@*)= (largo * ancho) > suficiente
>>> listo
False

>>> listo (*@\tikzmark{trmSetbool}@*)= (1.2*largo * ancho) > suficiente
>>> listo
True
\end{lstlisting}

\trmcolumns

\drawTikZComment[below right=2em and 3em of pic cs:trmSetbool]{trmSetbool}{\scriptsize Se puede asignar un resultado lógico}
\drawTikZAnotherArrow{bgnSetbool}{trmSetbool-comment}

## Tipos de dato lógico: {.fragile}

\simpleTitle{Probemos algunas expresiones compuestas:}

\begin{lstlisting}
>>> (1.2*largo * ancho) > suficiente and alto > 5
True

>>> listo and alto > 5
True

>>> listo = largo > 10 and ancho <= 6
>>> listo
False

>>> listo = largo > 10 or ancho <= 6
>>> listo
True

>>> not (largo > 10 or ancho <= 6)
False
\end{lstlisting}

## Funciones booleanas {.fragile}

- Una función también puede retornar un tipo lógico:

\begin{lstlisting}[style=frame02]
# esPar: int -> bool(*@\tikzmark{booltypedesign01}@*)
# Si 'numero' es par, retorna verdadero;
# en caso contrario retorna False.
# Ejemplo: esPar(7) retorna False
def esPar(numero):
    return (numero % 2) == 0

# Test
assert esPar(2) == True
assert esPar(11) == False
\end{lstlisting}

\pause

\drawTikZComment[right=8em of pic cs:booltypedesign01]{booltypedesign01}{\scriptsize Nuevo tipo de dato en nuestra receta}

## Funciones booleanas {.fragile}

- Una función también puede retornar un tipo lógico:

\begin{lstlisting}[style=frame02]
# esMultiploDe: int int -> bool
# Si 'numero' es múltiplo de 'factor', retorna verdadero;
# en caso contrario retorna False.
# Ejemplo: esMultiploDe(7, 3) retorna False
def esMultiploDe(numero, factor):
    return (numero % factor) == 0

# Test
assert esMultiploDe(21, 7) == True
assert esMultiploDe(18, 5) == False
\end{lstlisting}

