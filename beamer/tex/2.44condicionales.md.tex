## Condiciones múltiples {.fragile}

\simpleTitle{Otro ejemplo:}

\placeLogoTurnSignBig{keepstraight,turnleft,turnright}

\bgnblockgood[wd=.8\textwidth]

Dados dos números, determinar el mayor.

\alert{Destacar cuando sean iguales.}

\trmblockgood

\begin{lstlisting}[style=frame02]
# imprimeElMayorOIgual: int int -> None
# Recibe dos números, e indica en pantalla cuál es el mayor,
# o avisa cuando son iguales.
# Ejemplo: imprimeElMayorOIgual(11, 3)
def imprimeElMayorOIgual(numA, numB):
    if numA > numB:
        print('El mayor es: ' + str(numA))
    elif numA < numB:(*@\tikzmark{elifintroduction}@*)
        print('El mayor es: ' + str(numB))
    else:
        print('Los dos números son iguales.')
\end{lstlisting}

\drawTikZComment[right=10em of pic cs:elifintroduction, text width=30mm]{elifintroduction}{\scriptsize \nzinlinecode{elif} es como se escribe \nzinlinecode{else if}}
\vspace{-3ex}

## Condiciones múltiples {.fragile}

\placeLogoTurnSignSmall{keepstraight,turnleft,turnright}

\bgncolumns
\column{.52\textwidth}
\vspace{-1ex}

\simpleTitle{Código:}

\begin{lstlisting}
num1 = input('Ingrese un número: ')
num2 = input('Ingrese otro: ')
imprimeElMayorOIgual(num1, num2)
\end{lstlisting}

\column{.48\textwidth}
\vspace{-1ex}

\simpleTitle*{Ejecución 1:}

\begin{exampleConsole}
Ingrese un número: \codeInput{2}
Ingrese otro: \codeInput{7}
El mayor es: 7
\end{exampleConsole}

\fullrule

\simpleTitle*{Ejecución 2:}

\begin{exampleConsole}
Ingrese un número: \codeInput{4}
Ingrese otro: \codeInput{4}
Los dos números son iguales.
\end{exampleConsole}

\trmcolumns

## Condiciones: más ejemplos {.fragile}

\bgnblockgood
Dados \alert{tres} números, \alert{retornar} el mayor.
\trmblockgood

\begin{lstlisting}
>>> num1 = input('Ingrese un número: ')
>>> num2 = input('Ingrese otro: ')
>>> num3 = input('Ingrese otro: ')
>>> 
>>> mayor = obtenerEl(*@\tikzmark{markGetMaxNum}@*)Mayor(num1, num2, num3)
>>> 
>>> print('El número mayor es: ' + str(mayor))
\end{lstlisting}

\pause

\drawTikZComment[below=5em of pic cs:markGetMaxNum]{markGetMaxNum}{\scriptsize Hay que determinar el mayor, y devolverlo.}
\vspace{-3ex}

\bgncolumns

\column{.5\textwidth}
\column{.5\textwidth}

\footnotesize

- Comparar los tres números.
- Determinar el mayor.
- Entregarlo usando \nzinlinecode{return}.

\trmcolumns

## Condiciones: más ejemplos {.fragile}

\simpleTitle{Solución 1:}

\begin{lstlisting}[style=frame02]
# obtenerElMayor: int int int -> int
# Recibe tres números y retorna el mayor.
# Ejemplo: obtenerElMayor(11, 3, 14) retorna 14
def obtenerElMayor(num1, num2, num3):
    mayor = num1         # supondremos que el primero es el mayor

    if mayor < num2:     # ahora veremos si num2 le gana
        mayor = num2

    if mayor < num3:     # ahora veremos si num3 le gana
        mayor = num3

    return mayor

assert obtenerElMayor(2, 4, -3) == 4
assert obtenerElMayor(-2, -12, 0) == 0
\end{lstlisting}

## Condiciones: más ejemplos {.fragile}

\simpleTitle{Solución 2:}

\begin{lstlisting}[style=frame02]
# obtenerElMayor: int int int -> int
# Recibe tres números y retorna el mayor.
# Ejemplo: obtenerElMayor(11, 3, 14) retorna 14
def obtenerElMayor(num1, num2, num3):
    if num1 > num2 and num1 > num3:
        mayor = num1
    elif num2 > num3:
        mayor = num2
    else:
        mayor = num3

    return mayor

assert obtenerElMayor(2, 4, -3) == 4
assert obtenerElMayor(-2, -12, 0) == 0
\end{lstlisting}


## Condiciones: más ejemplos {.fragile}

\simpleTitle{Solución 3 (usando la función \nzinlinecode{max()} de Python):}

\begin{lstlisting}[style=frame02]
# obtenerElMayor: int int int -> int
# Recibe tres números y retorna el mayor.
# Ejemplo: obtenerElMayor(11, 3, 14) retorna 14
def obtenerElMayor(num1, num2, num3):
    return max(num1, max(num2, num3))

assert obtenerElMayor(2, 4, -3) == 4
assert obtenerElMayor(-2, -12, 0) == 0
\end{lstlisting}

\pause

\bgnblockidea
Ojo: La función \nzinlinecode{max()} recibe tantos parámetros como queramos darle:
\trmblockidea

\begin{lstlisting}[style=frame02]
def obtenerElMayor(num1, num2, num3):
    return max(num1, num2, num3)
\end{lstlisting}

## Condiciones: ejemplo más complejo {.fragile}

\bgnblockgood
Dados tres números, imprímalos de menor a mayor.
\trmblockgood

\pause

\begin{lstlisting}[style=frame02]
# imprimeEnOrden: int int int -> None
# Recibe tres números y los imprime en pantalla, de menor a mayor.
def imprimeEnOrden(num1, num2, num3):
    menor = min(num1, num2, num3)
    mayor = max(num1, num2, num3)
    medio = (num1 + num2 + num3) - menor - mayor

    print('El menor es' + str(menor))
    print('El medio es' + str(medio))
    print('El mayor es' + str(mayor))
\end{lstlisting}

## Condiciones: manipulando fechas {.fragile}

- A lo largo del curso iremos viendo varios ejemplos de manipulación de fechas.
- Veamos ahora dos funciones necesarias en estos casos.

\vfill

\bgnblockgood
Implemente una función que reciba un número de mes y retorne su cantidad de días.\newline
Si el número de mes es inválido, retorne \nzinlinecode{-1}, y suponga que no hay años bisiestos.
\trmblockgood

## Condiciones: manipulando fechas {.fragile}

\begin{lstlisting}[style=frame02]
# diasEnMes: int -> int
# Recibe un número de mes y retorna su cantidad de días.
# Si es inválido, retorna -1.
# Ejemplo: diasEnMes(3) retorna 31
def diasEnMes(mes):
    if mes == 1 or mes == 3 or mes == 5 or mes == 7 or mes == 8 or
            mes == 10 or mes == 12:
        dias = 31
    elif mes == 4 or mes == 6 or mes == 9 or mes == 11:
        dias = 30
    elif mes == 2:
        dias = 28
    else:
        dias = -1
    return dias
assert diasEnMes(8) == 31
assert diasEnMes(11) == 30
\end{lstlisting}

## Condiciones: años bisiestos {.fragile}

\bgnblockgood
Implemente una \alert{<3->}{función que} reciba un año y \alert{<3->}{determine si} es bisiesto.
\trmblockgood

\pause

\vfill

\bld{Se define año bisiesto como:}

> Aquel que es divisible por 400 o bien es divisible por 4 pero no por 100.

\pause

\vfill

\bgnblockidea
Esta función debe ser *booleana*.
\trmblockidea

## Condiciones: años bisiestos {.fragile}

\begin{lstlisting}[style=frame02]
# esBisiesto: int -> bool
# Recibe un año y retorna True si es bisiesto.
# Ejemplo: esBisiesto(2017) retorna False
def esBisiesto(year):
    if year % 400 == 0:
        return True
    elif year % 4 == 0 and year % 100 != 0:
        return True
    else:
        return False
assert esBisiesto(1992) == True
assert esBisiesto(1990) == False
\end{lstlisting}

\bgnblockalert
Ojo: este código puede ser escrito más resumido...
\trmblockalert

## Condiciones: años bisiestos {.fragile}

\bgnblockidea
Recuerden: \nzinlinecode{return} termina la ejecución de la función...
\trmblockidea

\vspace{-2ex}
\bgncolumns
\column{.7\textwidth}

\begin{lstlisting}[style=frame02]
def esBisiesto(year):
    if year % 400 == 0:
        return True
    elif year % 4 == 0 and year % 100 != 0:
        return True
    else:
        return False
\end{lstlisting}

\column{.3\textwidth}

\trmcolumns

\vspace{-2ex}
\bgncolumns
\column{.3\textwidth}

\column{.6\textwidth}

\begin{lstlisting}[style=frame02]
def esBisiesto(year):
    if year % 400 == 0:
        return True
    if year % 4 == 0 and year % 100 != 0:
        return True
    return False
\end{lstlisting}

\column{.1\textwidth}

\trmcolumns

## Condiciones: días en mes, completo {.fragile}

\bgnblockgood
Reimplementa ahora \nzinlinecode{diasEnMes}, pero incorporando la noción de año bisiesto.
\trmblockgood

\pause

\begin{lstlisting}[style=frame02]
def diasEnMes(mes, anno):
    if mes == 1 or mes == 3 or mes == 5 or mes == 7 or mes == 8 or
            mes == 10 or mes == 12:
        dias = 31
    elif mes == 4 or mes == 6 or mes == 9 or mes == 11:
        dias = 30
    elif mes == 2:
        dias = 28
        if esBisiesto(anno):
            dias = 29
    else:
        dias = -1
    return dias
\end{lstlisting}

