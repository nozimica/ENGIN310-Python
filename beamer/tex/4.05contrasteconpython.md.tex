## Contraste con Python {.fragile}

- Tanto con Python como con VBA podemos programar.
- Sin embargo, existen algunas diferencias que veremos ahora.

\pause

- Esto es lo que revisaremos:
    - Misceláneos
    - Variables
    - Funciones
    - Condicionales
    - Ciclos For

## Contraste con Python {.fragile}

\simpleTitle{Misceláneos:}

- Comentarios dentro del código:

\bgncolumns

\column{.45\textwidth}
\centering \structure{VBA:}
\begin{lstlisting}[style=vba]
    ' Esto es un comentario
    ' siempre deben comenzar
    ' con un apóstrofo
\end{lstlisting}

\column{.45\textwidth}
\centering \structure{Python:}
\begin{lstlisting}[style=frame03]
    # Sin embargo, en Python
    # los comentarios empiezan
    # con un gato.
\end{lstlisting}

\trmcolumns

## Contraste con Python {.fragile}

\simpleTitle{Variables:}

\bgncolumns

\column{.45\textwidth}
\centering \structure{VBA:}
\begin{lstlisting}[style=vba]
    Dim nombreVariable As tipoDeDato
    nombreVariable = valorInicial
\end{lstlisting}

\column{.45\textwidth}
\centering \structure{Python:}
\begin{lstlisting}[style=frame03]
    nombreVariable = valorInicial
\end{lstlisting}

\trmcolumns

\vfill

\simpleTitle[1ex]{Tipos de dato:}

- \structure{Boolean:} booleano
- \structure{String:} texto
- \structure{Integer:} entero simple
- \structure{Double:} un número real

\vfill

- \structure{Long:} un entero grande (muchos dígitos)
    - Aproximadamente más que $2.147.483.648$
- \structure{Single:} un número real (poca precisión)
- \structure{Decimal:} un número real (mucha precisión)

## Contraste con Python {.fragile}

\simpleTitle{Declaración vs Inicialización:}

\bgnblockdefinition
\bld{Declaración:} es la línea que comienza con \nzinlinecode{Dim} y que \itt{declara}
el nombre y el tipo de dato de la nueva variable.
\trmblockdefinition

\bgnblockdefinition
\bld{Inicialización:} es cuando se le da un valor inicial a la nueva variable.
\trmblockdefinition

\pause
\vfill

\simpleTitle[1ex]{Observaciones:}

- En Python sólo se hace la Inicialización.
    - La declaración se hace de manera automática.
- En VBA la declaración debe estar en una línea \bld{aparte} de la inicialización.
    - Esto no es así en VB.Net.

## Contraste con Python {.fragile}

\simpleTitle{Ejemplos de declaración de variables:}

\begin{lstlisting}[style=vba]
    Dim nombre As String
    Dim edad As Integer
    Dim precio As Double
    Dim estaListo As Boolean
\end{lstlisting}

\simpleTitle{Ejemplos de declaración e inicialización:}

\begin{lstlisting}[style=vba]
    Dim direccion As String
    direccion = "Diagonal Paraguay con Portugal."
    Dim puntos As Integer
    puntos = 3
    Dim profundidad As Double
    profundidad = 34.6
    Dim estaFuncionando As Boolean
    estaFuncionando = True
\end{lstlisting}

## Contraste con Python {.fragile}

\simpleTitle{Funciones:}

- He aquí otra diferencia entre VBA y Python:

\bgnblockdefinition
\bld{Función:} es un grupo de instrucciones que puede recibir uno o más parámetros,
pero que \bld{debe retornar} un valor.
\trmblockdefinition

\bgnblockdefinition
\bld{Subrutina:} Es igual a una función, salvo que \bld{nunca retorna} un valor.
\trmblockdefinition

- En Python no existe un símil de las subrutinas, pues en este caso también son
funciones.

## Contraste con Python {.fragile}

\simpleTitle{Funciones:}

\centering \structure{VBA:}
\begin{lstlisting}[style=vba]
Function calculaPitagoras(a As Double, b As Double) As Double
    calculaPitagoras = Math.sqr(a*a + b*b)
End Function
\end{lstlisting}

\vfill
\centering \structure{Python:}
\begin{lstlisting}[style=frame03]
import math

# calculaPitagoras: num num -> float
def calculaPitagoras(a, b):
    return math.sqrt(a*a + b*b)
\end{lstlisting}

## Contraste con Python {.fragile}

\simpleTitle{Subrutinas:}

\bgncolumns

\column{.55\textwidth}
\centering \structure{Sólo en VBA:}
\begin{lstlisting}[style=vba]
Sub ReiniciarJuego()
    Range("B5:D5").Select
    With Selection
        .Merge
        .Value = "Palabra:"
        .HorizontalAlignment = xlRight
        .Font.Bold = True
    End With
    Range("E5:H5").Select
    With Selection
        .Merge
        .Value = ""
        .HorizontalAlignment = xlLeft
        .Interior.Color = RGB(220, 220, 220)
        .Borders.LineStyle = xlContinuous
    End With
End Sub
\end{lstlisting}

\column{.4\textwidth}

\includegraphics[width=\linewidth]{img/vbaCaptura01.png}

\trmcolumns

## Contraste con Python {.fragile}

\simpleTitle{Condicionales:}

\bgncolumns

\column{.45\textwidth}
\centering \structure{VBA:}
\begin{lstlisting}[style=vba]
If Len(Range("E5").Value) > 0 Then
    Range("E5").Font.Italic = True
End If
\end{lstlisting}

\column{.45\textwidth}
\centering \structure{Python:}
\begin{lstlisting}[style=frame03]
if len(palabra) > 0:
    print(palabra)
\end{lstlisting}

\trmcolumns

\vfill

\simpleTitle[1ex]{Ciclos For:}

\bgncolumns

\column{.55\textwidth}
\centering \structure{VBA:}
\begin{lstlisting}[style=vba]
Dim palabra As String
palabra = Range("B2").Value
For i = 1 To Len(palabra)
    Cells(3 + i, 2).Value = Mid(palabra, i, 1)
Next
\end{lstlisting}

\column{.35\textwidth}
\centering \structure{Python:}
\begin{lstlisting}[style=frame03]
palabra = raw_input('')
for i in range(len(palabra)):
    print(palabra[i])
\end{lstlisting}

\trmcolumns

