## Repaso: funciones y creación de módulos {.fragile}

\simpleTitle{Ejemplo:}

\bgnblocknormal[justified]
\small
Ramiro es un vendedor de tarros de café de grano, que importa directamente desde Colombia. Su proveedor
le informa semanalmente el precio al que le venderá los tarros (en CLP).
Por su parte, Ramiro vende cada tarro con un 40\% de margen de ganancia, pero hace descuentos cuando
vende dos tarros (10\%), o cuando vende tres o más tarros (25\%).
\newline

Programe una función que dado el costo del café y la cantidad de tarros a vender, indique el precio unitario;
y otra función que dado el costo y los tarros vendidos tanto a precio unitario, como con 10\% dcto y con 25\% dcto,
calcule el total de ganancias.
\newline

Como Ramiro vende en CLP, será bueno utilizar la función \nzinlinecode{round()}.
\trmblocknormal


## Repaso: funciones y creación de módulos {.fragile}

\simpleTitle{Funciones:}

- Función que calcula el precio unitario:
    - \nzinlinecode{precioTarros(costo, cantidad)}

\vfill

- Función que calcula la ganancia:
    - \nzinlinecode{calculaGanancia(costo, v1, v2, v3)}

\vfill

- Funciones auxiliares para calcular precios según volumen.
    - \nzinlinecode{precio1Tarro(costo)}\tikzmark{markBgnFuncAuxs}
    - \nzinlinecode{precio2Tarro(costo)}
    - \nzinlinecode{precio3Tarro(costo)}\tikzmark{markTrmFuncAuxs}

\braceRightwardsLabel[brace/.style={xshift=2em}]{markBgnFuncAuxs}{markTrmFuncAuxs}{Son funciones opcionales}


## Repaso: funciones y creación de módulos {.fragile}

\includecode[lastline=20]{../scripts/ventas.py}{ventas.py}

## Repaso: funciones y creación de módulos {.fragile}

\includecode[firstline=22, lastline=42]{../scripts/ventas.py}{ventas.py}

## Repaso: funciones y creación de módulos {.fragile}

\includecode[firstline=44]{../scripts/ventas.py}{ventas.py}
