## Variables

- Corresponde a los datos involucrados en nuestro problema.

- Conviene darles un nombre representativo:
    - \coordText{<3->}{r1}{\texttt{edad}}{n1}{\texttt{int}}
    - \coordText{<3->}{r2}{\texttt{anhoNac}}{n2}{\texttt{int}}
    - \coordText{<3->}{r4}{\texttt{altura}}{n3}{\texttt{float} (¿o \texttt{int}?)}
    - \coordText{<3->}{r3}{\texttt{apellido}}{n3}{\texttt{str}}

\pause
\vspace{-.8ex}

- No es lo mismo manejar números que palabras.
    - Tampoco es lo mismo manejar enteros ($\mathbb{Z}$) que reales ($\mathbb{R}$).
    - Esto es lo que se llama \textcolor{blue}{\textbf{tipos de dato}}.


- ¿Cuáles serían los tipos de dato de los ejemplos de arriba?

## Tipos de dato

- Simplemente se refiere a sus características más básicas.
- Existen muchos tipos de dato. Por ahora sólo usaremos:
    \pause

    - \textcolor{blue}{\textbf{int}}: números enteros.
        - \texttt{... -3, -2, -1, 0, 1, 2, 3 ...}

    \pause

    - \textcolor{blue}{\textbf{float}}: números reales.
        - \texttt{24.73, 3.1415926, 2.71828, -63.4}

    \pause

    - \textcolor{blue}{\textbf{str}}: palabras, frases, texto.
        - \texttt{"Árbol", "Puedo escribir con espacios..."}
        - \texttt{'Árbol', 'Puedo escribir con espacios...'}

\vfill
