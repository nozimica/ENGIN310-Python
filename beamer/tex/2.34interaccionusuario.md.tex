## Interacción con el usuario {.fragile}

- Hasta ahora todos los datos los hemos escrito dentro de los programas.

- Habría que modificar el archivo \bld{.py} para modificar un dato de entrada.

\vfill

\bgnblockalert
¡Esto es demasiado engorroso!
\trmblockalert

\vfill
\pause

\bgnblockidea
Por ello es que existen los comandos\newline{}\nzinlinecode{input()} y \nzinlinecode{raw_input()}:
\trmblockidea

## Interacción con el usuario {.fragile}

\begin{lstlisting}
>>> apellido = input('Ingrese su apellido: ')
Ingrese su apellido: 'Bodoque'(*@\tikzmark{intConComillas}@*)
>>>
>>> edad = input('Ingrese su edad: ')
Ingrese su edad: 29(*@\tikzmark{intSinComillas}@*)
>>>
>>> print(apellido)
Bodoque
>>>
>>> print('Bienvenido ' + apellido)
Bienvenido Bodoque
>>>
>>> print(apellido + ' ya ha cumplido ' + str(*@\tikzmark{intNumStr}@*)(edad))
Bodoque ya ha cumplido 29
\end{lstlisting}

\visible{<2->}{%
\drawTikZComment[pos={right},len={7em}]{intConComillas}{\scriptsize Ojo con las comillas}
\drawTikZComment[pos={right},len={8em}, text width=42mm]{intSinComillas}{\scriptsize Acá no hay comillas: es número}
\drawTikZComment[pos={above right},len={2em and -0.5em}, text width=40mm]{intNumStr}{\scriptsize Recuerden: los números deben ser convertidos a string en este caso}
}

## Interacción con el usuario {.fragile}

\begin{lstlisting}
>>> nombre = raw_input('Ingrese su nombre: ')
Ingrese su nombre: Juan Carlos(*@\tikzmark{intSinComillasraw}@*)
>>>
>>> print('¡Otra vez ' + nombre + ' ' + apellido + '!')
¡Otra vez Juan Carlos Bodoque!
>>>
>>> edad = raw_input('Ingrese su edad: ')
Ingrese su edad: 29
>>>
>>> print(apellido + ' ya ha cumplido ' + ed(*@\tikzmark{intNumStrraw}@*)ad)
Bodoque ya ha cumplido 29

\end{lstlisting}

\visible{<2->}{%
\drawTikZComment[pos={right},len={6em}]{intSinComillasraw}{\scriptsize Acá no hay comillas: es \nzinlinecode{raw_input}}
\drawTikZComment[pos={above right},len={2em and -0.5em}, text width=40mm]{intNumStrraw}{\scriptsize No es número, aunque lo parezca}
}

\visible{<3->}{%
\bgnblockdefinition
\nzinlinecode{input()}: interpreta lo que se le entrega, tal cual lo hacemos en IDLE.\newline
\nzinlinecode{raw_input()}: todo lo recibe como si fuera texto.
\trmblockdefinition
}
