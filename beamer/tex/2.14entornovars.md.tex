## Alcance de una variable {.fragile} 

- Ya hemos visto que a las variables hay que darles un valor antes de usarlas.

\begin{lstlisting}[style=frame02]
>>> altura
Traceback (most recent call last):
  File "<stdin>", line 1, in <module>
NameError: name 'altura' is not defined
>>>
>>> altura = 188
>>> altura
188
\end{lstlisting}

\pause

- Si una variable ha sido definida fuera de una función:
    - ¿Se puede usar dentro de la función? \newline{}
        \only{<3->}{\alert{Depende}}
- Si fue definida dentro de la función:
    - ¿Se puede usar fuera de ella? \newline{}
        \only{<3->}{\alert{No}}

\pause

## Alcance de una variable {.fragile} 

- Si no existe ni adentro ni afuera, no se puede usar.

\begin{lstlisting}[style=frame02]
>>> def calcVolumen(largo, ancho):
        return altura * largo * ancho

>>> calcVolumen(23, 3)
Traceback (most recent call last):
  File "<stdin>", line 1, in <module>
  File "<stdin>", line 2, in calcVolumen
NameError: name 'altura' is not defined
\end{lstlisting}

- Pero de existir afuera, sí se puede usar en la función:

\begin{lstlisting}[style=frame02,linebackgroundcolor={\btLstHL{1}}]
>>> altura = 2
>>> def calcVolumen(largo, ancho):
        return altura * largo * ancho

>>> calcVolumen(4, 3)
24
\end{lstlisting}

## Alcance de una variable {.fragile} 

- Si sólo existe afuera, desde la función ven siempre el último valor:

\begin{lstlisting}[style=frame02,linebackgroundcolor={\btLstHL{1,7}}]
>>> altura = 2
>>> def calcVolumen(largo, ancho):
        return altura * largo * ancho

>>> calcVolumen(4, 3)
24
>>> altura = 4
>>> calcVolumen(4, 3)
48
\end{lstlisting}

## Alcance de una variable {.fragile} 

- Si existe tanto adentro como afuera de la función, tiene prioridad el valor local:

\begin{lstlisting}[style=frame02,linebackgroundcolor={\btLstHL{1,3,8}}]
>>> altura = 2
>>> def calcVolumen(largo, ancho):
        altura = 3
        return altura * largo * ancho

>>> calcVolumen(4, 3)
36
>>> altura = 4
>>> calcVolumen(4, 3)
36
\end{lstlisting}

## Alcance de una variable {.fragile} 

- Si una variable es parámetro de una función, significa:
    - Se considera como una variable local.
    - El valor de esa variable es lo que viene desde afuera.

\begin{lstlisting}[style=frame02]
>>> altura = 2
>>> largo = 3
>>> ancho = 4
>>> def calcVolumen(largo, ancho):
        return altura * largo * ancho

>>> calcVolumen(5, 6)
60
>>> largo = 10
>>> calcVolumen(5, 6)
60
>>> altura = 10
>>> calcVolumen(5, 6)
300
\end{lstlisting}

## Alcance de una variable {.fragile} 

- Pero ojo, una variable local de una función no se ve desde afuera.

\begin{lstlisting}
>>> varGlobal = 2
>>> def func1():
        varLocal = 4
        return varGlobal + varLocal

>>> func1()
6
>>> print(varGlobal)
2
>>> print(varLocal)
Traceback (most recent call last):
  File "<stdin>", line 1, in <module>
NameError: name 'varLocal' is not defined
\end{lstlisting}


## Alcance de una variable {.fragile} 

\bgnblockdefinition
\bld{Alcance de una variable:} es el o los lugares del programa donde una variable puede ser usada.
\trmblockdefinition

\vfill

\simpleTitle{Alcance de una variable en Python:}

- Si fue definida **fuera de una función**, será visible globalmente (\textcolor{structure}{variable global}).
- Si fue definida dentro de una función, será visible sólo dentro de ella (\textcolor{structure}{variable local}).
    - Si es un parámetro, también es local.
    - Al terminar la función, la variable local desaparece.
    - Si la variable tiene el mismo nombre que una que es global, sólo
    la ``reemplaza'' mientras se ejecuta la función.

