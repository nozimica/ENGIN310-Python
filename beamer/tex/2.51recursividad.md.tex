## Recursividad {.fragile}

\bgncolumns

\column{0.6\textwidth}

\vspace{-1em}

\bgnblockgood[wd=.8\textwidth,centered]
Maurits Cornelis Escher
\trmblockgood

\centering ``Drawing hands''

\column{0.4\textwidth}
\includegraphics[width=0.8\textwidth,valign=t]{img/DrawingHands.jpg}
\trmcolumns

- Artista holandés.
- Célebre por sus obras inspiradas en temas relacionados con la matemática.
- Jugó impecablemente con figuras imposibles, desafiando las dimensiones espaciales.

- La paradoja en esta obra: ¿cuál es la mano que comenzó el dibujo?

## Recursividad {.fragile}

\simpleTitle{La base de la recursividad está en la naturaleza:}

¿Qué es un árbol?

- Un árbol es una rama.
- De una rama salen varias ramas.
- De una rama salen varias ramas.
- De una rama salen varias ramas.
- De una rama salen varias ramas.
...
- De una rama sale una hoja o una flor.

\bgnblocknormal
¿En qué se asemeja esto a un río?
\trmblocknormal

## Recursividad {.fragile}

\simpleTitle{Definamos entonces recursividad:}

\pause

\bgnblockdefinition
Para entender la \structure{recursividad},

\pause

primero hay que entender la \structure{recursividad}.

\trmblockdefinition

\bgncolumns

\column{.7\textwidth}

\only{<5>}{%

\footnotesize
\begin{itemize}
\item Para resolver su duda llame al 12345.
\item Usted ha llamado al 12345. Para contestar sus dudas, corte y marque el 12345.
\item Usted ha llamado al 12345. Para contestar sus dudas, corte y marque el 12345.
\item Usted ha llamado al 12345. Para contestar sus dudas, corte y marque el 12345.
\end{itemize}

}

\column{.3\textwidth}

\only{<4->}{%
\includegraphics[width=\textwidth,valign=t]{img/manthinking-1f914.pdf}
}

\trmcolumns

## Recursividad {.fragile}

\simpleTitle{Veamos una definición más amable:}

\bgnblockdefinition
\flushleft \vspace{-2ex} Un proceso es \structure{recursivo} si:
\vspace{1ex}

1. Tiene un caso base.
1. Hay una regla principal que se llama a sí misma
    - \normalsize \vspace{.4ex} Pero en cada llamado se va reduciendo.
    - Hasta finalmente llegar al caso base.

\vspace{.5ex}
\trmblockdefinition

\vfill

\bgncolumns
\column{.33\textwidth}
\centering
\includegraphics[width=.6\textwidth,valign=t]{img/spiral.png}
\column{.33\textwidth}
\centering
\includegraphics[width=.6\textwidth,valign=t]{img/samsung-shell.png}
\column{.33\textwidth}
\centering
\includegraphics[width=.6\textwidth,valign=t]{img/coffee_v10_2615.pdf}
\trmcolumns

## Recursividad {.fragile}

\simpleTitle{Ejemplos:}

- Elevar a potencias

$$ 2^0 = 1 $$
$$ 2^n = 2 * 2^{n-1} $$

- Factorial

$$ 0! = 1 $$
$$ n! = n * (n-1)! $$

## Recursividad: aplicaciones {.fragile}

\bgnblockgood
\strongText{Ejemplo:} Imprimir un string $n$ veces.
\trmblockgood

\pause
\vspace{1ex}

\simpleTitle{Lo que sabemos hacer hasta ahora:}

\bgncolumns[-4ex]
\column{.6\textwidth}

- Imprimir un string 1 vez:

\column{.4\textwidth}

\begin{lstlisting}
def imprime(palabra):
    print(palabra)
\end{lstlisting}

\trmcolumns

\pause

\bgncolumns[-2ex]
\column{.6\textwidth}

- Para imprimirlo 2 veces, podríamos hacer:

\column{.4\textwidth}

\begin{lstlisting}
(*@\tikzmark{markBgnBadCode}@*)def imprime2Veces(palabra):
    print(palabra)
    print(palabra)
\end{lstlisting}

\trmcolumns

\bgncolumns[-2ex]
\column{.6\textwidth}

- Para imprimirlo 3 veces, podríamos hacer:

\column{.4\textwidth}

\begin{lstlisting}
def imprime2Veces(palabra):(*@\tikzmark{markMdlBadCode}@*)
    print(palabra)
    print(palabra)
    print(palabra)(*@\tikzmark{markTrmBadCode}@*)
\end{lstlisting}

\trmcolumns

\pause

\vspace{-2ex}
\bgnblockdanger
Esto último es muy mala idea: no es sustentable.
\trmblockdanger

\begin{tikzpicture}[remember picture, overlay]
  \node[fit=(markBgnBadCode) (markMdlBadCode) (markTrmBadCode), inner sep=0em] (enc) {};
  \draw[red, line width=.4em, line cap=round] (enc.north west) -- (enc.south east);
  \draw[red, line width=.4em, line cap=round] (enc.north east) -- (enc.south west);
\end{tikzpicture}

## Recursividad, en la práctica {.fragile}

\simpleTitle{Elevar a potencias:}

$$ 2^0 = 1 $$
$$ 2^n = 2 * 2^{n-1} $$

- Por lo que para $n \in \mathbb{N}$:

\begin{eqnarray*}
2^0 & = & 1 \\
2^1 & = & 2 * 2^0 = 2 \\
2^2 & = & 2 * 2^1 = 2 * (2 * 2^0) = 4 \\
2^3 & = & 2 * 2^2 = 2 * (2 * 2^1) = 2 * (2 * (2 * 2^0)) = 8\\
    & \ldots & \\
2^n & = & 2 * 2^{n-1} = 2 * (2 * 2^{n-2}) = \ldots
\end{eqnarray*}


## Recursividad {.fragile}

\simpleTitle{Elevar a potencias: ejemplo de desarrollo para $n = 4$}

\vspace{-3ex}

\bgncolumns

\column{.6\textwidth}

\begin{eqnarray*}
2^4 & = & 2 * 2^3 \tikzmark{bgnBraceReduction} \\
    & = & 2 * (2 * 2^2) \\
    & = & 2 * (2 * (2 * 2^1)) \\
    & = & 2 * (2 * (2 * (2 * 2^0)) \tikzmark{trmBraceReduction} \\
    & = & 2 * (2 * (2 * (2 * 1) \tikzmark{trmBraceBaseCase}) \\
    & = & 2 * (2 * (2 * 2)) \\
    & = & 2 * (2 * 4) \\
    & = & 2 * 8 \tikzmark{trmGathering} \\
2^4 & = & 16 \\
\end{eqnarray*}

\column{.3\textwidth}

\trmcolumns

\drawBrace[brace/.style={xshift=.3em}, decoration/.style={structure}, comment/.style={text width=12em}]{bgnBraceReduction}{trmBraceReduction}{En cada paso se llama recursivamente a la misma función, pero reducida.\newline \newline }
\drawBrace[brace/.style={xshift=1.3em}, decoration/.style={structure}]{trmBraceReduction}{trmBraceBaseCase}{El caso base.}
\drawBrace[brace/.style={xshift=.3em}, decoration/.style={structure}, comment/.style={text width=12em}]{trmBraceBaseCase}{trmGathering}{\newline \newline Recolección de resultados de las llamadas recursivas.}
