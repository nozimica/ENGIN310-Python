## Receta de diseño {.fragile}

- No basta con definir una función.
- ¡Es necesario que expliquemos qué hace!

\muestraRecetaDisenno{}

\placeLogoOpt{xshift=-5mm,yshift=-2mm}{img/memo-1f4dd.pdf}

\pause

\bgnblockdanger
Ojo: si no haces esto en las evaluaciones, tu nota bajará \textbf{considerablemente}.
\trmblockdanger

## Receta de diseño: explicar el propósito {.fragile}

\structure{Explicar el propósito de la función significa:}

1. Establecer el contrato:
    - Establecer el nombre de la función.
        - Escoger un nombre representativo.
    - Indicar su(s) entrada(s)
    - Indicar su(s) salidas(s)
1. Comentar qué hace la función.
1. Escribir su encabezado.

\vfill

\bgnblockdefinition
\bld{Contrato:} Corresponde al ``compromiso'' que una función tiene con el mundo exterior,
y que se manifiesta en cómo la pueden invocar (su nombre), sus entradas y sus salidas.
\trmblockdefinition

## Receta de diseño: explicar el propósito {.fragile}

\simpleTitle{1.- Contrato: ejemplos}

- Calcular el perímetro de un triángulo:

\vspace{-.5em}
\begin{lstlisting}[basicstyle=\normalsize\ttfamily]
# (*@\tikzmark{nombreA}@*)perimetroTriangulo(*@\tikzmark{nombreB}@*): (*@\tikzmark{entradaA}@*)num num num(*@\tikzmark{entradaB}@*) -> (*@\tikzmark{salidaA}@*)num(*@\tikzmark{salidaB}@*)
\end{lstlisting}

\braceDownwardsLabel[labelDistance=4mm, labelWidth=6em]{nombreB}{nombreA}{El nombre de la función}
\braceDownwardsLabel[labelDistance=4mm, labelWidth=6em]{entradaB}{entradaA}{Las entradas}
\braceDownwardsLabel[labelDistance=4mm, labelWidth=6em]{salidaB}{salidaA}{La salida}

\vspace{3em}

- Calcular el consumo de un vehículo, dado su rendimiento en [km/lt] y
la distancia recorrida [mi].

\vspace{-.5em}
\begin{lstlisting}[basicstyle=\normalsize\ttfamily]
# (*@\tikzmark{nombreA2}@*)consumoVehiculo(*@\tikzmark{nombreB2}@*): (*@\tikzmark{entradaA2}@*)num num(*@\tikzmark{entradaB2}@*) -> (*@\tikzmark{salidaA2}@*)float(*@\tikzmark{salidaB2}@*)
\end{lstlisting}

\braceDownwardsLabel[labelDistance=4mm, labelWidth=6em]{nombreA2}{nombreB2}{El nombre de la función}
\braceDownwardsLabel[labelDistance=4mm, labelWidth=6em]{entradaA2}{entradaB2}{Las entradas}
\braceDownwardsLabel[labelDistance=4mm, labelWidth=6em]{salidaA2}{salidaB2}{La salida}

## Receta de diseño: explicar el propósito {.fragile}

\simpleTitle{1.- Contrato: descripción de entradas y salidas}

Usaremos estas palabras clave al mencionar tanto las entradas como las salidas:

\vfill

\bgnblockdefinition

\begin{tabular}{lc}
    \toprule
    \bld{Tipo de dato}    & \bld{Palabra clave} \\
    \midrule
    \bld{Tipo entero}:    & \strongText{int} \\
    \bld{Tipo real}:      & \strongText{float} \\
    \bld{Tipo numérico}: (ya sea entero o real) & \strongText{num} \\
    \bld{Tipo texto}:     & \strongText{str} \\
    \bottomrule
\end{tabular}

\trmblockdefinition

\vfill

## Receta de diseño: explicar el propósito {.fragile}

\simpleTitle{2.- Comentar qué hace la función:}

\begin{lstlisting}[style=frame02,linebackgroundcolor={\btLstHL{2-3}}]
# consumoVehiculo: num num -> float
# Calcula el consumo de combustible en litros, para un vehículo
# dado su rendimiento [km/lt] y la distancia recorrida [mi].
def consumoVehiculo(rendKmLt, distMillas):
    distKm = mi2Km(distMillas)
    consumoLt = distKm / rendKmLt
    return consumoLt
\end{lstlisting}


## Receta de diseño: explicar el propósito {.fragile}

\simpleTitle{3.- Escribir su encabezado}

Al tener el contrato ya establecido, entonces podemos construir el encabezado de la función:

\begin{lstlisting}[style=frame02,linebackgroundcolor={\btLstHL{4}}]
# consumoVehiculo: num num -> float
# Calcula el consumo de combustible en litros, para un vehículo
# dado su rendimiento [km/lt] y la distancia recorrida [mi].
def consumoVehiculo(rendKmLt, distMillas):
    distKm = mi2Km(distMillas)
    consumoLt = distKm / rendKmLt
    return consumoLt
\end{lstlisting}

## Receta de diseño: explicar el propósito {.fragile}

\simpleTitle{3.- Escribir su encabezado}

\bgnblockdefinition
\bld{Encabezado de una función:} Es la primera línea del código que define una función, y
en ella se establece su nombre y sus parámetros.
\trmblockdefinition

\pause

\begin{lstlisting}[basicstyle=\normalsize\ttfamily,linewidth=.9\textwidth,xleftmargin=.1\textwidth]
# perimetroTriangulo: num num num -> num
(*@\tikzmark{defC}@*)def(*@\tikzmark{defD}@*) (*@\tikzmark{nombreC}@*)perimetroTriangulo(*@\tikzmark{nombreD}@*)((*@\tikzmark{entradaC}@*)ladoA, ladoB, ladoC(*@\tikzmark{entradaD}@*)):
\end{lstlisting}

\braceDownwardsLabel[labelDistance=4mm, labelWidth=5em]{defC}{defD}{Define el inicio de la función}
\braceDownwardsLabel[labelDistance=4mm, labelWidth=6em]{nombreC}{nombreD}{El nombre de la función}
\braceDownwardsLabel[labelDistance=4mm, labelWidth=8em]{entradaC}{entradaD}{Los parámetros}

## Receta de diseño: dar ejemplos de uso {.fragile}

\bgnblockgood[wd=.8\textwidth,centered]
    {\large \bld{Receta de diseño:}}
    \begin{enumerate}
        \item Explicar su propósito.
        \item \alert{Dar ejemplos de su uso.}
        \item Especificar el cuerpo de la función.
        \item Comprobar que hace lo correcto.
    \end{enumerate}
\trmblockgood

- Que aporten en el proceso de entender qué hace una función:

\begin{lstlisting}[style=frame02,linebackgroundcolor={\btLstHL{3}}]
# areaRectangulo: num num -> num
# Calcula el área de un rectángulo de medidas largo y ancho
# Ejemplo: areaRectangulo(5, 4) retorna 20
def areaRectangulo(largo, ancho):
\end{lstlisting}

## Receta de diseño: cuerpo de la función {.fragile}

\bgnblockgood[wd=.8\textwidth,centered]
    {\large \bld{Receta de diseño:}}
    \begin{enumerate}
        \item Explicar su propósito.
        \item Dar ejemplos de su uso.
        \item \alert{Especificar el cuerpo de la función.}
        \item Comprobar que hace lo correcto.
    \end{enumerate}
\trmblockgood

- Esto corresponde a la programación propiamente tal de la función.
- Son las instrucciones que llevan a cabo, fundamentalmente:
    - El proceso de los datos de entrada.
    - Que una vez listos, generan las salidas.

## Receta de diseño: cuerpo de la función {.fragile}

\simpleTitle{¿Cómo escribir el cuerpo de una función?}

- Una vez bien comprendido cuál es su propósito.
    - Definidas bien sus entradas y salidas.

- Pensamos en la manera de procesar la entrada.
    - Tener claros el o los cálculos que ello implica.
    - Aparte de los cálculos, quizás sea necesario otros pasos adicionales.

- Nunca perdiendo de vista que el objetivo es producir la salida esperada.
    - Puede ser el resultado de un simple o complejo cálculo matemático.
    - Puede ser el proceso de texto (que veremos más adelante).

## Receta de diseño: cuerpo de la función {.fragile}

\simpleTitle{Ejemplos:}

\begin{lstlisting}[style=frame02]
# consumoVehiculo: num num -> float
# Calcula el consumo de combustible en litros, para un vehículo
# dado su rendimiento [km/lt] y la distancia recorrida [mi].
# Ejemplo: consumoVehiculo(14, 24) retorna 2.7588754285714288
def consumoVehiculo(rendKmLt, distMillas):
    distKm = mi2Km(distMillas)
    consumoLt = distKm / rendKmLt
    return consumoLt
\end{lstlisting}

\bgnblockdefinition
\nzinlinecode{return} es la palabra clave que permite (dentro de una función)
entregar el resultado a quien la haya invocado. Después de ella, la función
termina.
\trmblockdefinition

## Receta de diseño: cuerpo de la función {.fragile}

\simpleTitle{Ejemplos:}

\begin{lstlisting}[style=frame02]
# perimetroTriangulo: num num num -> num
# Calcula el perímetro de un triángulo, recibiendo el valor de
# sus tres lados: ladoA, ladoB, ladoC.
# Ejemplo: perimetroTriangulo(3, 4, 5) retorna 12
def perimetroTriangulo(ladoA, ladoB, ladoC):
    return ladoA + ladoB + ladoC
\end{lstlisting}

- Y no nos olvidemos de los \nzinlinecode{str}:

\begin{lstlisting}[style=frame02]
# repetirPalabra: str num -> str
# Retorna un nuevo string que vale el contenido de palabra repetido
# tantas veces como valga veces.
# Ejemplo: repetirPalabra('pio ', 3) retorna 'pio pio pio '
def repetirPalabra(palabra, veces):
    return palabra * veces
\end{lstlisting}

## Receta de diseño: comprobar la función {.fragile}

\bgnblockgood[wd=.8\textwidth,centered]
    {\large \bld{Receta de diseño:}}
    \begin{enumerate}
        \item Explicar su propósito.
        \item Dar ejemplos de su uso.
        \item Especificar el cuerpo de la función.
        \item \alert{Comprobar que hace lo correcto.}
    \end{enumerate}
\trmblockgood

\bgnblockdefinition
\bld{Pruebas o \textit{testing}}: Consiste en una o varias aplicaciones de los datos de entrada a la función para verificar si ésta produce lo que es esperable.
\trmblockdefinition

## Receta de diseño: comprobar la función {.fragile}

- Hay que indicarle a Python que lo que estamos haciendo es una \bld{prueba}.

\bgnblockidea
Para esto, se usa la instrucción \nzinlinecode{assert}.
\trmblockidea

\pause

\bgnblockdefinition
Comportamiento de \nzinlinecode{assert}:
\begin{footnotesize}
\begin{tabular}{lp{0.4\textwidth}}
    \toprule
    \bld{Resultado de la prueba} & \bld{Reacción} \\
    \midrule
    La prueba sí funciona & \textcolor{structure}{No hace nada} \\
    La prueba falla & \textcolor{alert}{Se genera un error \nzinlinecode{AssertionError}} \\
    \bottomrule
\end{tabular}
\end{footnotesize}
\trmblockdefinition

\vfill

- Sólo se puede hacer pruebas con valores escritos \itt{a mano}.

## Receta de diseño: comprobar la función {.fragile}

- \nzinlinecode{assert}: cuando la prueba sí funciona, no hace nada:

\begin{lstlisting}
>>> assert perimetroTriangulo(5, 5, 8) == 18
\end{lstlisting}

- \nzinlinecode{assert}: cuando la prueba no funciona, genera un error:

\begin{lstlisting}
>>> assert perimetroTriangulo(5, 5, 8) == 17
Traceback (most recent call last):
  File "<stdin>", line 1, in <module>
AssertionError
\end{lstlisting}

\bgnblockidea
No te olvides de usar siempre dos signos igual (\nzinlinecode{==}) al utilizar \nzinlinecode{assert}.
\trmblockidea

## Receta de diseño: hela aquí en acción {.fragile}

\simpleTitle{Ejemplo 1:}

- La receta completa, aplicada al cálculo del perímetro de un triángulo:

\begin{lstlisting}[style=frame02]
# perimetroTriangulo: num num num -> num
# Calcula el perímetro de un triángulo, recibiendo el valor de
# sus tres lados: ladoA, ladoB, ladoC.
# Ejemplo: perimetroTriangulo(3, 4, 5) retorna 12
def perimetroTriangulo(ladoA, ladoB, ladoC):
    return ladoA + ladoB + ladoC
assert perimetroTriangulo(3, 4, 5) == 12
assert perimetroTriangulo(12, 15, 20) == 47
\end{lstlisting}

## Receta de diseño: hela aquí en acción {.fragile}

\simpleTitle{Ejemplo 2:}

- Aplicando la receta a nuestra función \ttt{mi2Km}:

\begin{lstlisting}[style=frame02]
# mi2Km: num -> float
# Transforma un valor desde millas a kilómetros.
# Ejemplo: mi2Km(6) retorna 9.656064
def mi2Km(distMillas):
    kmMi = 1.609344
    return distMillas * kmMi
# Test
assert mi2Km(6) == 9.656064
\end{lstlisting}

## Receta de diseño: hela aquí en acción {.fragile}

\simpleTitle{Ejemplo 3:}

- Aplicando la receta a nuestra función \ttt{areaCirculo}:

\begin{lstlisting}[style=frame02]
# areaCirculo: num -> float
# Calcula el área de un círculo, dado su radio.
# Ejemplo: areaCirculo(11) retorna 380.13239
def areaCirculo(radio):
    PI = 3.1415
    return PI * (radio ** 2)
# Test
assert areaCirculo(16) == 804.24704
\end{lstlisting}

## Receta de diseño: hela aquí en acción {.fragile}

\simpleTitle{Ejemplo 4:}

- Aplicando la receta a nuestra función \ttt{areaCirculo}:

\begin{lstlisting}[style=frame02]
# consumoVehiculo: num num -> float
# Calcula el consumo de combustible en litros, para un vehículo
# dado su rendimiento [km/lt] y la distancia recorrida [mi].
# Ejemplo: consumoVehiculo(14, 24) retorna 2.7588754285714288
def consumoVehiculo(rendKmLt, distMillas):
    distKm = mi2Km(distMillas)
    consumoLt = distKm / rendKmLt
    return consumoLt
assert consumoVehiculo(14, 24) == 2.7588754285714288
assert consumoVehiculo(12.4, 9) == 1.1680722580645162
\end{lstlisting}

