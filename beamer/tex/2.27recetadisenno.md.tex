## Receta de diseño {.fragile}

- No basta con definir una función.
- ¡Es necesario que expliquemos qué hace!

\bgnblock{Receta de diseño}

\vspace{-2ex}

- Qué es lo que hace.
- Cuál o cuáles son sus entradas.
- Cuál es su salida.
- Al menos un ejemplo.
- Comprobar que hace lo correcto.

\vspace{-2ex}

\trmblock

\placeLogoOpt{xshift=-5mm,yshift=-2mm}{img/memo-1f4dd.pdf}

\pause

\bgnblockdanger
Ojo: si no haces esto en las evaluaciones, tu nota bajará \textbf{considerablemente}.
\trmblockdanger

## Receta de diseño {.fragile}

\bgnblock{Receta de diseño}
\scriptsize

\vspace{-3ex}

- Qué es lo que hace.
- Cuál o cuáles son sus entradas.
- Cuál es su salida.
- Al menos un ejemplo.
- Comprobar que hace lo correcto.

\vspace{-3ex}
\trmblock


\begin{onlyenv}<1>
\begin{lstlisting}[style=frame01]
""" mi2Km: num -> float
Transforma un valor desde millas a kilómetros.
Ejemplo: mi2Km(6) retorna 9.656064
"""
def mi2Km(distMillas):
    kmMi = 1.609344
    return distMillas * kmMi
# Test
assert mi2Km(6) == 9.656064
\end{lstlisting}
\end{onlyenv}

\begin{onlyenv}<2>
\begin{lstlisting}[style=frame01]
""" areaCirculo: num -> float
Calcula el área de un círculo, dado su radio.
Ejemplo: areaCirculo(11) retorna 380.13239
"""
def areaCirculo(radio):
    PI = 3.1415
    return PI * (radio ** 2)
# Test
assert areaCirculo(16) == 804.24704
\end{lstlisting}
\end{onlyenv}

\placeLogoOpt{xshift=-5mm,yshift=-2mm}{img/memo-1f4dd.pdf}


## Realizando cálculos (cont.) {.fragile}

\vspace*{-3ex}
\bgncolumns

\column{0.4\textwidth}

\bgnblocknormal
Para este código definimos una nueva función: \texttt{consumoVehiculo}
\trmblocknormal

\column{0.05\textwidth}
\column{0.55\textwidth}

\begin{lstlisting}
>>> rendKmLt = 14
>>> distMillas = 24
>>> distKm = mi2Km(distMillas)
>>> consumoLt = distKm / rendKmLt
\end{lstlisting}

\trmcolumns

\vspace{2em}
\begin{lstlisting}[style=frame01]
>>> """ consumoVehiculo: num, num -> float
>>> Calcula el consumo de combustible en litros, para un vehículo
>>> dado su rendimiento [km/lt] y la distancia recorrida [mi].
>>> Ejemplo: consumoVehiculo(14, 24) retorna 2.7588754285714288
>>> """
>>> def consumoVehiculo(rendKmLt, distMillas):
>>>     distKm = mi2Km(distMillas)
>>>     consumoLt = distKm / rendKmLt
>>>     return consumoLt
>>> assert consumoVehiculo(14, 24) == 2.7588754285714288
>>> assert consumoVehiculo(17, 12) == 1.1360075294117649
\end{lstlisting}

