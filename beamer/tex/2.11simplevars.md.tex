## Manejo de variables en Python {.fragile}

\vspace{-3ex}

\bgncolumns
\column{0.5\textwidth}

\simpleTitle{Definiendo variables}

\begin{lstlisting}
>>> a = 4
>>> a
4
>>> a = 7
>>> b = 11
>>> r = 2.1
>>> a
7
>>> print(b)
11
>>> a + r
9.1
>>> z = a + 3 * b
>>> z
40
\end{lstlisting}

\column{0.5\textwidth}

\simpleTitle{Varias variables en línea}

\begin{lstlisting}
>>> m, n = 234, 64
>>> m
234
>>> n
64
>>> m, n
(234, 64)
>>> print(m, n)
(234, 64)
>>>
>>> m2, n2 = m / 10.0, n**0.5
>>> m2, n2
(23.4, 8.0)
\end{lstlisting}
\trmcolumns

## Manejo de variables en Python {.fragile}

\vspace{-3ex}

\bgncolumns

\column{0.5\textwidth}

\simpleTitle{Intercambio de valores}

\begin{lstlisting}
>>> a = 10
>>> b = -5
>>> print(a, b)
(10, -5)
>>> 
>>> aux = a
>>> a = b
>>> b = aux
>>> 
>>> print(a, b)
(-5, 10)
\end{lstlisting}

\column{0.5\textwidth}

\simpleTitle{Pero miren esto:}

\begin{lstlisting}
>>> a = 23
>>> b = -12
>>> b, a = a, b
>>> a,b
(-12, 23)
\end{lstlisting}

\trmcolumns

## Variables: malas y buenas prácticas {.fragile}

\vspace{-2ex}

\bgncolumns

\column{0.3\textwidth}

\begin{lstlisting}
>>> x = 4
>>> y = 6
>>> z = x * y
>>>
>>> z
24
\end{lstlisting}
\column{0.4\textwidth}

\pause
\vspace{1ex}
\includegraphics[width=.7\textwidth,valign=t]{img/658-emoji_android_fearful_face.png}
\column{0.3\textwidth}

\begin{footnotesize}
    \begin{nzalertblock}
    ¡No se sabe qué \textbf{representan} estas variables!
    \end{nzalertblock}

    \begin{nzalertblock}
    \footnotesize No es buena idea usar sólo letras.
    \end{nzalertblock}
\end{footnotesize}

\trmcolumns


\pause
\vspace{2ex}

\bgncolumns

\column{0.5\textwidth}
\begin{lstlisting}
>>> ancho = 4
>>> largo = 6
>>> area = ancho * largo
>>>
>>> area
24
\end{lstlisting}
\column{0.5\textwidth}
\pause

\bgnblocknormal
    Ahora sí sabemos qué está pasando...
\trmblocknormal

\bgnblocknormal
    \footnotesize Los nombres de las variables son autoexplicativos.
\trmblocknormal

\trmcolumns

## Variables: repaso

Repasemos el concepto de *Variable*:

\vfill

\bgnblockdefinition
\bld{Variable:} Corresponde a un valor que puede cambiar a lo largo de la ejecución de un algoritmo.
\trmblockdefinition

\bgncolumns
\column{.65\textwidth}

- Se identifican con un nombre único y permanente.
    - \underline{Conviene}: usar \bld{nombres adecuados} (evitará futuras confusiones).
    - \underline{Recomendado}: que el nombre \bld{comience con} una letra \bld{minúscula}.

\column{0.35\textwidth}
\vspace{2ex}
\small
\tikz[baseline=(exp1.north)]{
    \node[coordinate] (exp1) {};
    \node[exampleboxblue,below=3mm of exp1] (exp2) { \texttt{total = 12} };
    \node[exampleboxblue,below of=exp2] (exp3) { \texttt{total = -2} };
    \node[exampleboxblue,below of=exp3] (exp4) { \texttt{total = i} };
    \node[exampleboxgreen,right=14mm of exp1] (exp9) { \texttt{i = 0} };
    \node[exampleboxgreen,below of=exp9] (exp8) { \texttt{i = 2} };
    \node[exampleboxgreen,below of=exp8] (exp7) { \texttt{i = i + 1} };
}
\trmcolumns

## Ejemplos

\begin{block}{Calcular lado de un triángulo rectángulo}
Usando tres variables para representar a los lados de un triángulo rectángulo, del
cual conocemos sólo la hipotenusa y otro de los lados, calcular el valor
del tercer lado.
\end{block}

\begin{block}{Calcular dígitos}
Dado un número de 5 dígitos, obtener el primer y el último dígito.
\end{block}
