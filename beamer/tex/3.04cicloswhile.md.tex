## Ciclos while {.fragile}

- Permiten repetir una serie de instrucciones, una cantidad arbitraria de veces.
    - Esa cantidad puede estar definida \itt{a priori}.
    - O puede ser determinada en "tiempo real".

\pause

\bgnblockdefinition
\nzinlinecode{while condicion:} \newline
    \nzinlinecode{repito instrucciones}
\trmblockdefinition

## Ciclos while {.fragile}

\simpleTitle{Comparación entre \nzinlinecode{for} y \nzinlinecode{while}}

\bgncolumns[-2ex]
\column{.45\textwidth}

- Analicemos el siguiente código:

\begin{lstlisting}[style=frame03]
>>> for j in range(10):
...     print(j)
... 
0
1
2
3
4
5
6
7
8
9
\end{lstlisting}

\column{.45\textwidth}

- Y veamos cómo se hace lo mismo con \nzinlinecode{while}:

\begin{lstlisting}[style=frame03]
>>> j = 0
>>> while j < 10:
...     print(j)
...     j = j + 1
... 
0
1
2
3
4
5
6
7
8
9
\end{lstlisting}

\trmcolumns

## Ciclos while {.fragile}

- Esta es otra manera de hacer el código anterior:

\begin{lstlisting}[style=frame03]
desde = 0
hasta = 10

while desde < hasta:(*@\tikzmark{syntaxWhile}@*)
    print(desde)
    desde = desde + 1(*@\tikzmark{incrIdx}@*)
\end{lstlisting}

\drawTikZComment[pos={right},len={6em},text width=56mm]{syntaxWhile}{\scriptsize mientras se dé la condición, repito las instrucciones dentro del bloque}
\drawTikZComment[pos={below right},len={.6em and 6em},text width=56mm]{incrIdx}{\scriptsize \structure{esto es clave:} asegurarse de que la condición termine alguna vez}

\vfill

\bgnblockdanger
Pero: ¿para qué usar \nzinlinecode{while} de esta manera, si hace lo mismo que \nzinlinecode{for}?
\trmblockdanger

## Ciclos while {.fragile}

\simpleTitle{Ventajas de \nzinlinecode{while}}

- Esta herramienta tiene la gran ventaja de ser mucho más flexible que \nzinlinecode{for}.
    - No se limita a recorrer un arreglo ya lleno de elementos.

\bgnblockidea
Analicemos una aplicación práctica: llenar un arreglo inicialmente vacío, con datos entregados
por el usuario.
\trmblockidea

\vfill

\begin{lstlisting}[style=frame03]
# creo un arreglo vacío
# le solicito números al usuario
# mientras no me entregue un ""
#    lo recibido lo convierto a número y lo agrego al arreglo
#    vuelvo a solicitar números al usuario
# después de que recibí "", aplico la función que calcula la desviación estándar
\end{lstlisting}

## Ciclos while: desviación estándar v2 {.fragile}

\simpleTitle{Calcular la desviación estándar de un conjunto de valores:}

- Se procesará un conjunto de valores que serán ingresados sucesivamente.
- No se sabe en un principio cuántos serán.
    - Recibiremos un string vacío (\ttt{"}\ttt{"}) cuando no haya más datos.
- Para este conjunto de valores, debemos calcular su desviación estándar.

## Ciclos while: desviación estándar v2 {.fragile}

\nzincludecode{../scripts/e10while01.utf8.py}{Solicitar números al usuario}

## Ciclos while: raíces cuadradas v2 {.fragile}

- Recordemos el cálculo de raíces cuadradas, usando el\newline ``método de Herón'':

$$ \sqrt{x} = \lim_{n \to \infty} r_{n+1} \qquad \text{tal que:}  \begin{cases} r_{n+1} & = \frac{1}{2} \left(r_n + \frac{x}{r_n}\right) \\ r_0 & = \text{\footnotesize{número arbitrario positivo}} \\ \end{cases} $$

\vfill
\pause

\bgnblockidea
¿Qué tal si lo implementamos usando \nzinlinecode{while}?
\trmblockidea

## Ciclos while: raíces cuadradas v2 {.fragile}

- Primero que todo, volver a incluir la función \nzinlinecode{cerca}:

\nzincludecode[firstline=1,lastline=8]{../scripts/e11raizheronwhile.utf8.py}{Definir función cerca}

## Ciclos while: raíces cuadradas v2 {.fragile}

\nzincludecode[firstline=10]{../scripts/e11raizheronwhile.utf8.py}{Implementar cálculo con ciclos}

## Ciclos while: raíces cuadradas v2 {.fragile}

- Comparemos la versión recursiva con esta nueva versión:

\nzincludecode[firstline=10,lastline=20]{../scripts/e06raizheron.utf8.py}{}

\nzincludecode[firstline=10,lastline=17]{../scripts/e11raizheronwhile.utf8.py}{}

## Ciclos while: búsquedas del tesoro {.fragile}

- Recordemos al famoso y malévolo corsario Barba Azul.
- Vamos a reimplementar las 5 búsquedas que realizamos sobre el listado de cartas
con el camino al tesoro.

\pause

- ¿Cómo quedan los datos ahora?

\pause

\begin{lstlisting}[style=frame03]
import estructura
estructura.crear("carta", "lugar referencia profundidad")

cartas = []

carta1 = carta("isla gaviota", "detrás de la roca", "2 metros")
cartas = cartas + [carta1]

carta2 = carta("playa larga", "al terminar el sendero", "3 metros")
cartas = cartas + [carta2]

carta3 = carta("arrecife escondido", "al medio", "4 metros")
cartas = cartas + [carta3]

print("Arreglo:", cartas)
\end{lstlisting}


## Ciclos while: búsquedas del tesoro {.fragile}

\simpleTitle{Preguntas útiles de responder:}

1. ¿Hay alguna carta que nos lleve a la ``isla gaviota''?
1. ¿Cuántas cartas nos harán excavar 4 metros de profundidad?
1. Haga una nueva lista, pero esquivando ``playa larga''.
1. ¿Cuántos lugares distintos hay en toda la lista?
1. ¿Cuál es la profundidad más repetida en toda la lista?

## Ciclos while: buscar un valor {.fragile}

\simpleTitle{1.- ¿Hay alguna carta que nos lleve a la ``isla gaviota''?}

\vspace{1ex}
\bld{Estrategia de resolución:}

- Recorro todo el arreglo de cartas:
    - Para cada carta, sólo si su atributo \nzinlinecode{lugar} es igual a lo buscado, retorno \ttt{True}.
    - Ojo: si se retorna al interior de un ciclo, se termina tanto el ciclo como la función.
- Si completé el recorrido de la lista, y no encontré nada, retorno \ttt{False}.

## Ciclos while: buscar un valor {.fragile}

\simpleTitle{1.- ¿Hay alguna carta que nos lleve a la ``isla gaviota''?}

\begin{lstlisting}[style=frame02]
def buscarLugar(arrCartas, buscado):
    # variable que irá recorriendo cada índice
    i = 0
    # cantidad de cartas (límite del recorrido)
    n = len(arrCartas)
    # mientras queden cartas por recorrer
    while i < n:
        # carta actual
        estaCarta = arrCartas[i]
        # si encontré el lugar, retorno True
        if estaCarta.lugar == buscado:
            return True
        i = i + 1                       # incremento el índice
    # si llegué hasta acá, la búsqueda no resultó...
    return False
\end{lstlisting}

## Ciclos while: buscar y contar {.fragile}

\simpleTitle{2.- ¿Cuántas cartas nos harán excavar 4 metros de profundidad?}

\vspace{1ex}
\bld{Estrategia de resolución:}

- Dado que el resultado es contar algo, defino una variable (contador) para ello.
    - Inicialmente en 0.
- Recorro el arreglo, y para cada elemento:
    - Cada vez que su atributo \ttt{profundidad} sea lo buscado:
        - Incremento el contador.
- Finalizado el recorrido, retornamos nuestro contador.

## Ciclos while: buscar y contar {.fragile}

\simpleTitle{2.- ¿Cuántas cartas nos harán excavar 4 metros de profundidad?}

\begin{lstlisting}[style=frame02]
def contarProfundidad(arrCartas, profBusc):
    i = 0               # variable que irá recorriendo cada índice
    n = len(arrCartas)  # cantidad de cartas (límite del recorrido)
    total = 0           # contador

    # mientras queden cartas por recorrer
    while i < n:
        estaCarta = arrCartas[i]        # carta actual
        # si encontré la profundidad, incremento el contador
        if estaCarta.profundidad == profBusc:
            total = total + 1
        i = i + 1                       # incremento el índice
    # si llegué hasta acá, la búsqueda terminó
    return total
\end{lstlisting}

## Ciclos while: buscar y descartar {.fragile}

\simpleTitle{3.- Haga una nueva lista, pero esquivando ``playa larga''}

\vspace{1ex}
\bld{Estrategia de resolución:}

- Se crea un arreglo vacío.
- Recorro el arreglo, y para cada elemento:
    - Si su atributo \ttt{lugar} \alert{NO} es lo buscado.
        - Agrego esta carta al nuevo arreglo (la copia).
- Terminado el recorrido, retorno el nuevo arreglo.

\bgnblockidea
\bld{Ojo:} vamos a crear una copia del arreglo
original, con la salvedad que cada vez que encontremos el lugar buscado,
no lo copiamos (nos lo saltamos).
\trmblockidea

## Ciclos while: buscar y descartar {.fragile}

\simpleTitle{3.- Haga una nueva lista, pero esquivando ``playa larga''}

\footnotesize Se presenta una solución genérica, donde el segundo parámetro es el \ttt{lugar} a esquivar.

\begin{lstlisting}[style=frame02]
def esquivarLugar(arrCartas, buscado):
    i = 0               # variable que irá recorriendo cada índice
    n = len(arrCartas)  # cantidad de cartas (límite del recorrido)
    nvoArr = []         # arreglo a retornar

    # mientras queden cartas por recorrer
    while i < n:
        estaCarta = arrCartas[i]        # carta actual
        # si esta carta no está en el lugar buscado, la agrego
        if estaCarta.lugar != buscado:
            nvoArr = nvoArr + [estaCarta]
        i = i + 1                       # incremento el índice
    # si llegué hasta acá, la búsqueda terminó
    return nvoArr
\end{lstlisting}


## Ciclos while: contar ocurrencias {.fragile}

\simpleTitle{4.- ¿Cuántos lugares distintos hay en toda la lista?}

\vspace{1ex}
\bld{Estrategia de resolución:}

- Se crea un arreglo vacío:
    - Lo llamaremos \ttt{lugaresUnicos}.
    - Este arreglo irá guardando las cartas cuyas ciudades aún no estén en este
    mismo arreglo.
- Recorro el arreglo original, y para cada elemento:
    - Si su atributo \ttt{lugar} no está aún en \ttt{lugaresUnicos}:
        - Agrego esta carta al nuevo arreglo (la copia).
- Terminado el recorrido, retorno la cantidad de elementos de \ttt{lugaresUnicos}.

\bgnblockidea
Para cada lugar encontrado, sólo se considera la primera vez que se encuentra.
\trmblockidea

## Ciclos while: contar ocurrencias {.fragile}

\simpleTitle{4.- ¿Cuántos lugares distintos hay en toda la lista?}

\begin{lstlisting}[style=frame02]
def contarLugaresDistintos(arrCartas):
    i = 0               # variable que irá recorriendo cada índice
    n = len(arrCartas)  # cantidad de cartas (límite del recorrido)
    lugaresUnicos = []  # arreglo que guardará los lugares

    # mientras queden cartas por recorrer
    while i < n:
        estaCarta = arrCartas[i]        # carta actual
        # si este lugar no está en el arreglo lugaresUnicos
        if not buscarLugar(lugaresUnicos, estaCarta.lugar):
            lugaresUnicos = lugaresUnicos + [estaCarta]
        i = i + 1                       # incremento el índice
    # si llegué hasta acá, la búsqueda terminó
    # retorno la cantidad de elementos de lugaresUnicos
    return len(lugaresUnicos)
\end{lstlisting}


