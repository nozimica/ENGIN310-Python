## Archivos de texto {.fragile}

- Los archivos de texto nos permiten guardar datos.

- Es la manera más simple de hacerlo.
    - Se pueden leer desde cualquier lenguaje de programación.
    - Podemos usar programas sencillos para ver su contenido.

- Con los archivos de texto hay fundamentalmente dos operaciones:
    - Abrir y escribir.
    - Abrir y leer.

## Archivos de texto {.fragile}

\simpleTitle{Crear un archivo:}

\begin{lstlisting}[style=frame02]
# -*- coding: cp1252 -*-
with open('archivo.txt', 'w') as puntero:
    puntero.write("Estamos aprendiendo a escribir archivos.")
    puntero.write("")
    puntero.write("Final.")
\end{lstlisting}

\pause

- Su contenido será éste:

\begin{exampleFile}
Estamos aprendiendo a escribir archivos.Final.
\end{exampleFile}

\pause
\vfill

\bgnblockdanger
¡Nos faltó crear \bld{saltos de línea}!
\trmblockdanger


## Archivos de texto {.fragile}

\simpleTitle{Crear un archivo:}

\begin{lstlisting}[style=frame02]
# -*- coding: cp1252 -*-
with open('archivo.txt', 'w') as puntero:
    puntero.write("Estamos aprendiendo a escribir archivos.\n")
    puntero.write("\n")
    puntero.write("Final.\n")
\end{lstlisting}

\pause

- Ahora su contenido será éste:

\begin{exampleFile}
Estamos aprendiendo a escribir archivos.

Final.
\end{exampleFile}

## Archivos de texto {.fragile}

\simpleTitle{Leer un archivo:}

\begin{lstlisting}[style=frame02]
# -*- coding: cp1252 -*-
with open('archivo.txt', 'r') as puntero:
    linea = puntero.readline()
    print(linea)
\end{lstlisting}

- Lo que nos imprimirá en pantalla la primera línea:

\begin{exampleFile}
Estamos aprendiendo a escribir archivos.

\end{exampleFile}

\bgnblockdanger
He aquí un gran problema: ¿qué hacemos cuando no sabemos cuántas líneas tiene
el archivo?.
\trmblockdanger

## Archivos de texto {.fragile}

\simpleTitle{Leer un archivo:}

\bgnblockdanger
He aquí un gran problema: ¿qué hacemos cuando no sabemos cuántas líneas tiene
el archivo?.
\trmblockdanger

\pause

- Aún cuando sepamos la cantidad de líneas, si son muchas, tendremos que repetir el \nzinlinecode{readline}
muchas veces.

\pause

\bgnblockidea
Para leer archivos sin tener que saber la cantidad de líneas que tiene, podemos
utilizar un \nzinlinecode{for}.
\trmblockidea

## Archivos de texto {.fragile}

\simpleTitle{Leer un archivo:}

\begin{lstlisting}[style=frame02]
# -*- coding: cp1252 -*-
with open('archivo.txt', 'r') as puntero:
    for linea in puntero:
        print(linea)
\end{lstlisting}

- Lo que nos imprimirá en pantalla algo distinto a lo esperado:

\begin{exampleFile}
Estamos aprendiendo a escribir archivos.



Final.

\end{exampleFile}

## Archivos de texto {.fragile}

\simpleTitle{Leer un archivo:}

- Al leer línea por línea, leemos esa línea \bld{junto} al salto de línea.
    - Tenemos que sacarle ese carácter.

\bgnblockidea
Para ello, usamos la función \nzinlinecode{strip} de los strings:
\trmblockidea

- \nzinlinecode{strip}: Quita los caracteres en blanco y saltos de línea tanto al inicio como al final.
    - También quita otros caracteres, tales como tabuladores.

## Archivos de texto {.fragile}

\simpleTitle{Leer un archivo:}

\begin{lstlisting}[style=frame02]
# -*- coding: cp1252 -*-
with open('archivo.txt', 'r') as puntero:
    for linea in puntero:
        # le sacamos los caracteres en blanco y saltos de línea
        # tanto al inicio como al final
        linea = linea.strip()
        print(linea)
\end{lstlisting}

- Lo que nos imprimirá en pantalla el contenido del archivo:

\begin{exampleFile}
Estamos aprendiendo a escribir archivos.

Final.
\end{exampleFile}

## Archivos de texto {.fragile}

\simpleTitle{Leer un archivo: mayor control sobre las líneas}

- He aquí un caso de uso muy típico:

\bgnblockidea
¿Qué pasa cuando quiero descartar la primera línea, porque es el encabezado?
\trmblockidea

\pause

- La respuesta:

\bgnblockdefinition
Los archivos tienen la función \nzinlinecode{next()} que permite pasar, literalmente,
a la línea siguiente.
OJO: esta función retorna el texto de la línea saltada.
\trmblockdefinition

## Archivos de texto {.fragile}

\simpleTitle{Leer un archivo: mayor control sobre las líneas}

- Sea este archivo de texto:

\begin{exampleFile}
CODIGO:Nombre:Cantidad
SDF32:Linterna:23
RTK22:Pantalla:4
TYN85:Ventilador:0
\end{exampleFile}

- Para leer su contenido, descartando el encabezado:

\begin{lstlisting}[style=frame02,linebackgroundcolor={\btLstHL{3}}]
# -*- coding: cp1252 -*-
with open('archivo.txt', 'r') as puntero:
    puntero.next()
    for linea in puntero:
        linea = linea.strip()
        print(linea)
\end{lstlisting}


## Archivos de texto {.fragile}

\simpleTitle{Función \nzinlinecode{open}}
    
- Esta función toma dos parámetros:
    - El nombre del archivo.
    - Un indicador del tipo de operación:

\bgnblockgood

- \ttt{'w'}: escribir archivo, creándolo o sobreescribiéndolo.
- \ttt{'r'}: leer archivo, que de no existir, lanza un error.

\trmblockgood

