## Strings {.fragile}

- Ya conocemos los \ttt{string}.
    - Sabemos cómo concatenarlos.\tikzmark{strconcat}
    - Sabemos cómo "multiplicarlos".\tikzmark{strmult}

\drawTikZComment[pos={above right},len={.6em and 5em},text width=10mm]{strconcat}{\ttt{+}}
\drawTikZComment[pos={below right},len={.6em and 5em},text width=10mm]{strmult}{\ttt{*}}

- ¿Qué más se podría hacer con ellos?

\bgnblockidea
¡Muchas cosas! Los \ttt{string} son mucho más que simples variables.
\trmblockidea

## Strings {.fragile}

\bgnblockdefinition
Se puede entender un \ttt{string} como 0 ó más caracteres, uno después del otro: casi
como un \structure{arreglo} de caracteres.
\trmblockdefinition

\simpleTitle[2ex]{Los \ttt{string} se comportan a veces, como arreglos:}

\bgncolumns[-2ex]
\column{.45\textwidth}

\begin{lstlisting}[style=frame03]
>>> rioCercano = "Mapocho"
>>> rioLejano = "Aconcagua"
>>> lugarAlEste = "Quebrada de Macul"

>>> rioCercano[2:5]
'poc'
>>> rioCercano[3:]
'ocho'
>>> rioLejano[::-1]
'augacnocA'
>>> lugarAlEste[6:15]
'da de Mac'
\end{lstlisting}

\column{.45\textwidth}

\begin{lstlisting}[style=frame03]
>>> len(rioCercano)
7
>>> len(lugarAlEste)
17
>>> for x in lugarAlEste[12:]:
...   print(x)
... 
M
a
c
u
l
\end{lstlisting}

\trmcolumns

## Strings {.fragile}

\simpleTitle{Y a veces se comportan como números:}

\bld{Comparación:}

- Tal como los números, los \ttt{string} tienen un orden: todo carácter está "a la izquierda"
o "a la derecha" de todo otro carácter.

- Ojo: entre los tipos de caracteres existe este orden:

\bgnblockdefinition
\ttt{espacio < dígitos < mayúsculas < minúsculas}
\trmblockdefinition


## Strings {.fragile}

\simpleTitle{Y a veces se comportan como números:}

- Se puede usar las funciones \nzinlinecode{min} y \nzinlinecode{max}.

\begin{lstlisting}[style=frame03]
>>> if 'A' < rioCercano < 'Z':
...     print("Está entre la A y la Z")
... 
Está entre la A y la Z

>>> min(rioCercano, rioLejano)
'Aconcagua'
>>> max(rioCercano, max(rioLejano, lugarAlEste))
'Quebrada de Macul'
\end{lstlisting}

## Strings: funciones propias {.fragile}

\simpleTitle{Cambiar el ``case'' (mayúsculas o minúsculas)}

\begin{lstlisting}[style=frame03]
>>> rioMasLejano = "Itata"
>>> rioMasLejano.upper()
'ITATA'
>>> rioMasLejano.lower()
'itata'
\end{lstlisting}

\simpleTitle[1ex]{\nzinlinecode{find}: Encontrar un carácter dentro de un \ttt{string}}

- Puedo indicarle, opcionalmente, desde qué posición buscar (y hasta dónde buscar).

\bgncolumns[-1ex]
\column{.45\textwidth}

\begin{lstlisting}[style=frame03]
>>> rioMasLejano.find('t')
1
>>> rioMasLejano.find('t', 2)
3
>>> rioMasLejano.find('t', 4)
-1
\end{lstlisting}

\column{.45\textwidth}

\begin{lstlisting}[style=frame03]
>>> rioMasLejano.find('a')
2
>>> rioMasLejano.find('a', 0)
2
>>> rioMasLejano.find('a', 0, 2)
-1
\end{lstlisting}

\trmcolumns


## Strings: funciones propias {.fragile}

\simpleTitle{\nzinlinecode{split}: Dividir un string según un carácter}

- Una tarea muy frecuente es tomar un string separar todas sus palabras, metiéndolas a un arreglo.

\begin{lstlisting}[style=frame03]
>>> texto = "El barco navega frente al puerto"
>>> texto.split()
['El', 'barco', 'navega', 'frente', 'al', 'puerto']
>>> texto.split(' ')
['El', 'barco', 'navega', 'frente', 'al', 'puerto']
>>> texto.split('e')
['El barco nav', 'ga fr', 'nt', ' al pu', 'rto']
\end{lstlisting}

- \nzinlinecode{split} recibe un string que usará como separador. Ese texto desaparece de las
palabras que quedan dentro del arreglo.
- Opcionalmente puede recibir un segundo parámetro \nzinlinecode{int}, que indica la cantidad
máxima de separaciones que realiza.

## Strings: funciones propias {.fragile}

\simpleTitle{\nzinlinecode{sort}: Ordena los elementos de un arreglo}

- Los ordena de menor a mayor, modificando el arreglo.

\begin{lstlisting}[style=frame03]
>>> texto = "El barco navega frente al puerto"
>>> palabras = texto.split()
>>> palabras = texto.split(' ')
>>> palabras
['El', 'barco', 'navega', 'frente', 'al', 'puerto']
>>> palabras.sort()
>>> palabras
['El', 'al', 'barco', 'frente', 'navega', 'puerto']
\end{lstlisting}

## Strings: operadores {.fragile}

- Veremos tres operadores de strings:
    - Concatenación: signo \ttt{+}
    - Repetición: signo \ttt{*}
    - Substring: \ttt{in}

\simpleTitle{Operador substring}

- Permite saber si un string está contenido dentro de otro

\begin{lstlisting}[style=frame03]
>>> texto = "El barco navega frente al puerto"
>>> 'barco' in texto
True
>>> 'Barco' in texto
False
>>> 'el' in texto
False
>>> 'a fr' in texto
True
\end{lstlisting}

