## Conceptos importantes

- Los algoritmos a medida que avanzan, van cambiando en su estado. Esto se ve reflejado en el valor de las variables que contiene.
- Siempre se debe tener en cuenta cómo se afectan los valores de las variables que determinan la ejecución de un algoritmo.
- Al diseñar un algoritmo, se deben tener en cuenta todas las condiciones posibles que pueden ocurrir y manejarlas adecuadamente.

## Estrategia de resolución

- Una de las cosas más importantes es comprender el problema que se debe resolver.
    - Una forma de hacerlo es resolver a mano algunos casos, con datos específicos, para entender todo el problema.

\pause
\vfill

- Siempre es útil desmenuzar el problema completo en pequeños subproblemas y resolverlos por sí solos. 
    - Lo que se debe desarrollar es la habilidad de identificar los \textbf{subproblemas}.

\pause
\vfill

- Este método se conoce como:
      \hspace{20mm}\begin{varblock}[.8\textwidth]{}
        \centering  \textbf{Dividir para Reinar}.
      \end{varblock}

## Dividir para reinar

- Lo que se debe desarrollar es la habilidad de identificar los subproblemas  que componen el problema original.
- Una vez completada la resolución de aquellos subproblemas, podemos armar la solución definitiva.
- Es más cómodo y eficiente resolver varios problemas pequeños, que uno solo pero mucho más grande.
- Hay que saber cuándo dejar de subdividir.


