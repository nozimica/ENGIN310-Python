## Funciones {.fragile}

\bgnenviron{<1>}{onlyenv}

- Es bueno tomar nota de algunos cuidados que debemos tener con las funciones:

\trmenviron{onlyenv}

\bgnblocknormal

\begin{enumerate}
    \item<1-| alert@2> Como todo en Python, distinguen mayúsculas de minúsculas.
    \item<1-| alert@3> Debe dárseles siempre los argumentos que necesita.
\end{enumerate}

\trmblocknormal

\pause

\bgnenviron{<2>}{onlyenv}
\bgncolumns
\column{.3\textwidth}

\begin{lstlisting}
>>> mi2Km(6)
9.656064
>>> mi2Km(0.3)
0.4828032
>>> mi2Km(12)
19.312128
\end{lstlisting}

\column{.7\textwidth}

\begin{lstlisting}
>>> mk2km(12)
Traceback (most recent call last):
  File "<stdin>", line 1, in <module>
  NameError: name 'mk2km' is not defined
>>>
>>> Mk2km(12)
Traceback (most recent call last):
  File "<stdin>", line 1, in <module>
  NameError: name 'Mk2km' is not defined
\end{lstlisting}

\trmcolumns
\trmenviron{onlyenv}

\bgnenviron{<3>}{onlyenv}
\bgncolumns
\column{.3\textwidth}

\begin{lstlisting}
>>> mi2Km(6)
9.656064
>>> mi2Km(0.3)
0.4828032
>>> mi2Km(12)
19.312128
\end{lstlisting}

\column{.7\textwidth}

\begin{lstlisting}
>>> mi2Km()
Traceback (most recent call last):
  File "<stdin>", line 1, in <module>
  TypeError: mi2Km() missing 1 required positional argument: 'distMillas'
>>>
>>> mi2Km(12, 33)
Traceback (most recent call last):
  File "<stdin>", line 1, in <module>
  TypeError: mi2Km() takes 1 positional argument but 2 were given
\end{lstlisting}

\trmcolumns
\trmenviron{onlyenv}

## Indentación {.fragile}

- Es bueno que hablemos del concepto de \strongText{bloque}:

\bgnblockdefinition
\bld{Bloque:} Es un grupo de instrucciones que pertenecen a un mismo contexto.
\trmblockdefinition

- Ya conocemos uno de esos contextos: las \strongText{funciones}.

## Indentación {.fragile}

- Y al escribir en Python, ¿cómo indicamos que esas instrucciones pertenecen al mismo bloque?
    - Escribiéndolas con la misma \strongText{indentación}.

\pause

\bgnblockdefinition
\bld{Indentación:} Cantidad de espacios en blanco a la izquierda que tiene el inicio de una instrucción.
\trmblockdefinition

\vspace{2ex}

\pause

\bgnblockidea
Todo el cuerpo de la función debe estar \strongText{indentado}.
\trmblockidea


## Indentación {.fragile}

- Algunos errores posibles:

\begin{lstlisting}
>>> def mi2Km(distMillas):
... kmMi = 1.609344
  File "<stdin>", line 2
    kmMi = 1.609344
       ^
IndentationError: expected an indented block
>>>
>>> def mi2Km(distMillas):
...     kmMi = 1.609344
...    return distMillas * kmMi
  File "<stdin>", line 3
    return distMillas * kmMi
                           ^
IndentationError: unindent does not match any outer indentation level
\end{lstlisting}

## Ejercicios {.allowframebreaks}

1. Implementa una función que dado un el precio neto de un producto, entregue su precio con IVA.
1. Implementa una función que calcule el volumen de una esfera de radio $r$.
    - Recuerda que la fórmula es: $\frac{4}{3}\pi r^3$.
1. Implementa una función que realice el siguiente cálculo: $\frac{a+b^2}{3c}$
1. Calcula la distancia (en 2 dimensiones) entre dos puntos. Considera que cada punto se representa
como dos valores: $x_i$ e $y_i$.
1. Para cualquier número de exactamente 2 dígitos, calcula la suma de sus dígitos.
1. Para cualquier número de exactamente 3 dígitos, calcula la suma de sus dígitos.
1. Supongamos que hoy estamos a domingo:
    - ¿cuántos días faltarán para el siguiente domingo dentro de 20 días?
    - ¿cuántos días faltarán para el siguiente domingo dentro de 80 días?
1. Considerando el problema anterior:
    1. ¿cómo lo implementarías en una función?
    1. ¿cuáles serían su(s) entrada(s), constante(s) y salida(s)?
1. Dado un triángulo de lados $a$, $b$ y $c$, escribe una función que calcule su semiperímetro. Este
valor corresponde a la mitad del perímetro.
1. La \textsc{fórmula de Herón} nos permite calcular el área de un triángulo sólo usando
el largo de sus tres lados ($a$, $b$ y $c$).
La fórmula es: $$ area = \sqrt{s (s-a) (s-b) (s-c)} $$ donde $s$ es el semiperímetro.
Escriba una función para calcular el área de un triángulo usando esta fórmula.
1. Haga una función que calcule la propina a pagar, dado un $precio$ y un $porcentajePropina$.
Debe usar \nzinlinecode{print} para desplegar primero el valor de la $propina$, y después
el valor del $precio + propina$. Retorne también esta última suma.
